\chapter{Conclusiones}
\label{chp:conclusions}

One improvement is the expansion of the training dataset to 12.7 million samples, from the previous 2.3 million samples. These new samples are categorized into 19 different video content types, which improves over the previous 2 categories. Rangerx also implements a new feature extraction strategy, which improves detection performance to almost 0 Out Of Bag (OOB) prediction error over the complete dataset. Finally, Rangerx improves over the previous RDF Detector by optimizing the feature extraction and modifying the RDF backend to the Ranger C++ Core Library.

Las conclusiones no son un resumen de lo realizado sino a lo que ha llevado el
desarrollo de la tesis, no perdiendo de vista los objetivos planteados desde
el principio y los resultados obtenidos.  En otras palabras, qué se concluye o
a qué se ha llegado después de realizado la tesis de maestría.  Un error
común es ``concluir'' aspectos que no se desarrollaron en la tesis, como
observaciones o afirmaciones derivadas de la teoría directamente.  Esto último
debe evitarse.

Es fundamental en este capítulo hacer énfasis y puntualizar los
aportes específicos del trabajo.

Es usual concluir con lo que queda por hacer, o sugerencias para mejorar los
resultados.

