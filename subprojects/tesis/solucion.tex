\chapter{The Rangerx Artifact Detection System}
\label{ch:solucion}

\section{Artifact Detection System Structure}
\label{sec:sol_struct}

The Artifact Detection System generates detection masks from frames with artifacts. These masks are white on macroblocks which contain artifacts and black on the remaining artifacts. Figure \ref{fig:detector_overview} illustrates the input frame with artifacts, which is processed by the system, and finally produces the detection masks.

\begin{figure} [!h]
  \centering
  
  \includegraphics{detector_general}
  
  \caption{General Artifact Detection System Overview. }
  \label{fig:detector_overview}

\end{figure}

In order to achieve this, the Artifact Detection System must first extract features from the input frame. These features become the inputs to the RDF Classifier. The RDF Classifier outputs a prediction vector. The prediction vector is then mapped to the binary mask output. Figure \ref{fig:detector_steps} illustrates these steps.

\begin{figure} [!h]
  \centering
  
  \includegraphics{detector_steps}
  
  \caption{Artifact Detection process. }
  \label{fig:detector_steps}

\end{figure}

This structure was used in the dispTEC2-2022 RDF system. The Rangerx system maintains this structure and improves over the previous system by 

\subsection{The Feature Extractor}
\label{sec:sol_features}

The feature extractor

\begin{figure} [!h]
  \centering
  
  \includegraphics{extractor_steps}
  
  \caption{Feature extraction process. }
  \label{fig:extractor_steps}

\end{figure}

\subsection{The RDF Classifier}
\label{sec:sol_rdf}

\subsection{The Mask Generatorr}
\label{sec:sol_maskgen}

\section{Training}
\label{sec:sol_}



\section{Testing}
\label{sec:intro_problem}

\section{Video Artifact Detection}
\label{sec:intro_problem}

\section{Video Artifact Detection}
\label{sec:intro_problem}

\section{Video Artifact Detection}
\label{sec:intro_problem}
