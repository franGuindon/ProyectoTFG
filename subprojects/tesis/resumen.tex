\chapter*{Resumen}
\thispagestyle{empty}

Las herramientas de trasmisión de video en vivo son habituales en contextos profesionales, académicos y recreativos gracias a los avances actuales en velocidad de red, procesamiento de imágenes y tecnología digital. Existe un gran interés por garantizar una alta calidad de experiencia (QoE) para estos servicios. Estos servicios pueden sufrir de artefactos de vídeo causados por pérdidas de paquetes durante la transmisión de vídeo. El proyecto anterior de dispTEC2-2022 pretendía crear un Sistemas de Restauración de Vídeo para restaurar eficazmente el vídeo con pérdidas en línea, pero concluyó sin un Sistema de Detección de Artefactos adecuado. Este proyecto propone Rangerx, que mejora el sistema de detección de dispTEC2-2022, basado en Bosques de Decisión Aleatoria (RDF), en 3 aspectos: Rangerx fue entrenado con un conjunto de datos más grande, con una nueva estrategia optimizada de extracción de características, y con un nuevo RDF optimizado.
\bigskip

%% Defina las palabras clave con defKeywords en config.tex:
\textbf{Palabras clave:} \thesisKeywords

\clearpage
\chapter*{Abstract}
\thispagestyle{empty}

Current advances in network speed, image processing, and digital technology have rendendered live video tools commonplace in professional, academic, and recreational contexts. Therefore, there has been increased interest in ensuring a high quality of experience (QoE) for these services. These services may suffer from video artifacts caused by packet losses during video transmission. The previous dispTEC2-2022 project aimed to create a Video Restoration System to efficiently restore video with losses on-line, but concluded without a proper Artifact Detection System. This project proposes Rangerx, which improves in 3 ways over the dispTEC2-2022 Random Decision Forest (RDF) artifact detection approach: Rangerx was trained with a larger dataset, with a new optimized feature extraction strategy, and with a new optimized RDF backend.

\bigskip

\textbf{Keywords:} word 1, word 2, 

\cleardoublepage

%%% Local Variables: 
%%% mode: latex
%%% TeX-master: "main"
%%% End: 
