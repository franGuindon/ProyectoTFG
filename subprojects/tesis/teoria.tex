\chapter{Theoretical Framework}
\label{ch:marco}

\section{The Amdahl Virtual Machine}
\label{sec:theory_amdahl}
  
The School of Electronic Engineering disposes of a large server for students, professors, and researchers. Within this server, resources can be allocated in a variety of ways to suit different performance needs. Amdahl is a virtual machine that runs within the school's server. Amdahl emulates an x86 32 core machine with 62.8Gb of memory available to the main running process. Table \ref{tab:cpu-info} shows Amdahl's CPU information, in particular the speed. Amdahl provides an optimized CPU environment to train the Random Forest classifiers developed in this project. Amdahl was capable of loading and training with a complete dataset of 19 sets of 200 720p frames, this generated in total 12745200 training examples.

\begin{table}[htbp]
  \centering
  \begin{tabular}{|c|c|}
    \hline
    \textbf{Model Name} & Intel(R) Xeon(R) CPU E5-4667 v3 @ 2.00GHz \\
    \hline
    \textbf{CPU MHz} & 1995.379 \\
    \hline
  \end{tabular}
  \caption{Amdahl's CPU Information. All 32 cores are identical.}
  \label{tab:cpu-info}
\end{table}

\section{Theoretical Framework Outline}

\begin{enumerate}
\item Video artifacts
  \begin{enumerate}
  \item h264 encoding
    \begin{enumerate}
    \item Who created it?
    \item What is the goal?
    \item What are the characteristics of h264 encoding
      \begin{enumerate}
      \item What are macroblocks?
      \item Why are we using macroblocks to delimit the binary masks
      \end{enumerate}
    \end{enumerate}
  \item video transmission
  \item compression artifacts
  \item packet loss artifacts
  \end{enumerate}
\item Random Decision Forests
  \begin{enumerate}
  \item Classical Machine Learning
  \item Feature extraction
    \begin{enumerate}
    \item Why is it done and why is it important?
    \item How is feature extraction done?
    \item What to keep in mind for feature extraction?
    \end{enumerate}
  \item Metrics
    \begin{enumerate}
    \item OOB
    \item Confusion Matrix
    \item Precision and Recall
    \end{enumerate}
  \end{enumerate}
\end{enumerate}

\section{Definición de marco de trabajo a utilizar}

\section{Definición de características de entrenamiento a seleccionar}

Tipos de artefactos en MPEG-2 según \cite{Greengrass2009}.

- definición de artefacto de video

Slice Error:
- when network drops at least one IP packet within an I-, P- or B- frame
- causes strips of the video to appear distorted

Blocking or Pixelization:
- when loss happens within a reference frame (either an I- or a P- frame)

Ghosting:
- Loss in I-frames near scene changes. Parts of the previous sceen still appear.
- New P- and B- frames continue to spread the ghosting effect

Freeze Frame:
- loosing a stream of MPEG frames will freeze the current frame for a while

Loss in I-Frames:
Loss in I-Frames affect a complete GoP and is only recovered until the next I-frame.
Loss in the beginning of an I-Frame that looses header information results in pixelization throughout the whole GoP. Same effect as loosing the whole I-frame.
Loss within I-frame looses slices and affect the whole GoP.
Since 2 B-frames before I frame reference it in 15:2, they are also affected.
Thus, loss in I-frames affect 17 frames assuming a 15:2 group of pictures (GoP)
Longer GoP means longer artefacts if they occur in I-frames, but they allow
higher compression, which allows to transmit higher-quality video with a given
bit rate.

Loss in P-Frames:
Since P-frames reference previous frames, loss affects from P-frame onward in a GoP.
Loosing the P-frame header is as significant as lossing the whole P-frame, and affects the remainder of the GoP.
Loss withing P-frame results in slice errors that continue through GoP.

Tipos de artefactos en MPEG / H26x según \cite{Glavota2016}.

Dos categorías: Artefactos rectangulares y artefactos irregulares, causados por propagación de error.

Non-concealed Artifact (NCA): Un artefacto rectangular donde la región perdida aparece en negro y tiene bordes rectangules bien definidos.

Low spatial activity artifact (LSAA): Un artefacto rectangular que no es negro y difiere de sus alrededores y tiene algo de actividad espacial.
High spatial activity artifact (HSAA): El artefacto contiene detalle y bordes internos.
Temporally-concealed artifacts (TCA): Un rango de artefactos donde los valores de los macrobloques contiene valores repetidos de cuadros anteriores.

Propagation artifacts (PA): Los errores irregulares suceden por la propagación de errores rectangulares en cuadros predictivos. Estos pueden tener formas distorcionadas y valores de pixel que cambian significativamente

Filtro pasa altas con coefficientes [1, -1] aplicado tanto en componentes de luminancia como en componentes de chrominancia en direcciones horizontales y verticales,

Los bordes de NCA y LSAA y hasta HSAA dan valores altos de gradiente.
Los TCA y PA dependen de muchas variables y son complicados de predecir.

Selección de características:

Filtro pasa altas para detección de bordes:
- Filtro [1, -1]
- Filtro [1/2, 0, -1/2] (diferencias centrales)
- Filtro de Sobel [1, 0, -1; 2 0 -2; 1 0 -1], \cite{Korhonen2018}
- Filtro de Prewitt [1, 0, -1, 1, 0, -1, 1, 0, -1]
- Filtro de Roberts cross [1 0, 0 -1]
...
- Filtro de Canny (caro)
- Filtro de Deriche (caro)
- Filtro differencial (asume suavizamiento gaussiano, caro)

Sugerencias de \cite{Glavota2016, Korhonen2018} es que se puede medir el promedio y desviación estandar de la detección de bordes a nivel de macrobloque.

Posiblemente vamos a querer realizar este cálculo a un nivel mayor que el de un solo macrobloque para incluir algo de contexto.

Promedio y Varianza de macrobloque y vecinos
Promedio y Varianza de bloques
Promedios y Varianza de bordicidad ([1, -1])

Histograma de imagen
Definir probabilidad a partir del cual se considera error

% \import{~/Documents/2023_S1/Proyecto/tools/cpp_tools/trainer/}{training_theory}

