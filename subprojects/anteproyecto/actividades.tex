%% ---------------------------------------------------------------------------
%% intro.tex
%%
%% Introduction
%%
%% $Id: intro.tex 1477 2010-07-28 21:34:43Z palvarado $
%% ---------------------------------------------------------------------------

\chapter{Actividades}
\label{chp:actividades}

Para entrenar y optimizar un bosque de decisión aleatoria que realice detección de artefactos de videos, es necesario iniciar con tareas de investigación y preparación del conjunto de datos. Es necesario investigar las características de los artefactos de video y de los métodos de extracción de características para entrenar bosques de decisión aleatoria. Además es necesario investigar los marcos de trabajo bajo los cuales se desarrollan y optimizan bosques de decisión aleatoria. Se deben investigar las posibles optimizaciones en el hardware a utilizar, pero la mayoría de las tareas de optimización dependen de primero entrenar un bosque que realice la detección de artefactos.

La Tabla \ref{tab:actividades} resume las actividades a desarrollar, además de sus requisitos y los objetivos a los cuales corresponden.

\begin{table} [!h]
    \caption{Actividades a realizar, organizadas por objetivo e indicando sus respectivos requisitos y duraciones (en días)}
    \label{tab:actividades}
    \centering
    \begin{tabular}{p{0.4 cm} p{7.8 cm} p{1.6 cm} p{2 cm} p{2 cm}}
        \hline
        $\#$ & Actividad & Objetivo & Requisitos & Duración \\
        \hline
        1 & Coleccionar videos a utilizar como conjunto de datos & 1 & - & 2 \\
        2 & Simular pérdida de paquetes sobre conjunto de datos & 1 & 1 & 1 \\
        3 & Generar datos de referencia comparando datos con pérdidas y datos originales & 1 & 2 & 2 \\
        \hline
        4 & Investigar características de artefactos de video por pérdida de paquetes & 2 & - & 10 \\
        5 & Investigar estrategias de extracción de características para bosques de decisión aleatoria orientados a imágenes & 2 & - & 10 \\
        6 & Investigar marcos de trabajo para entrenar bosques de decisión aleatoria & 2 & - & 10 \\
        7 & Determinar estrategias de extracción de características sobre datos con pérdidas & 2 & 4, 5 & 5 \\
        8 & Determinar e instalar marco de trabajo para entrenar bósques de decisión aleatoria & 2 & 6 & 5 \\
        9 & Entrenar bosques & 2 & 3, 7, 8 & 20 \\
        \hline
        10 & Investigar estrategias de optimización por software para estrategias de extracción de características & 3 & 5 & 25 \\
        11 & Investigar estrategias de optimización por hardware para la Jetson TX2 & 3 & - & 25 \\
        12 & Implementar estrategias de optimización sobre extracción de características & 3 & 9, 10, 11 & 20 \\
        \hline
    \end{tabular}
\end{table}
