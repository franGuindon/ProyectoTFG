\documentclass[12pt,oneside]{book}
\usepackage[letterpaper, total={19cm, 20cm}]{geometry}

\usepackage{graphicx}
\usepackage{csquotes}
\usepackage[backend=biber,              % Use biber/biblatex
            style=ieee,
            sorting=none,
            citestyle=numeric-comp]{biblatex}
\usepackage{array, multirow}
\usepackage{caption}
\usepackage{subcaption}

\addbibresource{../tesis/literatura.bib}
\usepackage{listings}
\usepackage{xcolor}

\definecolor{codegreen}{rgb}{0,0.6,0}
\definecolor{codegray}{rgb}{0.5,0.5,0.5}
\definecolor{codepurple}{rgb}{0.58,0,0.82}
\definecolor{backcolour}{rgb}{0.95,0.95,0.92}

\lstdefinestyle{mystyle}{
    backgroundcolor=\color{backcolour},   
    commentstyle=\color{codegreen},
    keywordstyle=\color{magenta},
    numberstyle=\tiny\color{codegray},
    stringstyle=\color{codepurple},
    basicstyle=\ttfamily\footnotesize,
    breakatwhitespace=false,         
    breaklines=true,                 
    captionpos=b,                    
    keepspaces=true,                 
    numbers=left,                    
    numbersep=5pt,                  
    showspaces=false,                
    showstringspaces=false,
    showtabs=false,                  
    tabsize=2,
    mathescape=true
}

\lstset{style=mystyle}

% TODO: Add documentation and change name of doc to the effect of .bash_aliases

\usepackage{enumitem} % nested enumerations



\begin{document}
 \graphicspath{{./}{../tesis/fig/}}
  Tecnológico de Costa Rica
  \par\vspace{1mm}
  Escuela de Ingeniería Electrónica
  \par\vspace{1mm}
  Programa de Licenciatura en Ingeniería Electrónica
  \par\vspace{10mm}
  Trabajo Final de Graduación
  \par\vspace{1mm}
  Francis Guindon Badilla
  \par\vspace{1mm}
  2018259419
  \par\vspace{10mm}
  \large\textbf{Bitácora - Semana 7}
  \par\vspace{10mm}
  \small

  \begin{table} [!h]
    \centering
    \small
    \begin{tabular}{p{1.5 cm} p{2.1 cm} p{5 cm} p{8 cm}}
      \hline
      Fecha & Duración & Actividad & Descripción \\
      \hline
      22/3/23 & 1 h & Reunión de avance 7 & Metas son: Perfilar herramientas, Iniciar introducción de tesis, Continuar con extracción de features \\
      22/3/23 & 4 h & Extracción de features & Se solucionó pulga en cálculo de filtrado \\
      23/3/23 & 4 h & & Se completó caso sencillo de extracción de características para un solo borde \\
      24/3/23 & 4 h & & Se optimizó caso sencillo usando macros y loop unrolling \\
      25/3/23 & 4 h & & Se solucionó pulgas en cálculos de extracción de características comparando con resultados de python \\
      27/3/23 & 4 h & & Se empezó a modularizar código en varias funciones para facilitar mantenimiento \\
      28/3/23 & 2 h & Perfilado & Se perfiló código de python y c++ para permitir comparar \\
      28/3/23 & 2 h & Tesis & Se creó un esquema de la introducción \\
      \hline
      \textbf{Total} & 21 h \\
      \hline
    \end{tabular}
  \end{table}
  
  \vfill

  \begin{tabular}{p{3 cm} p{10 cm}}
    Firma profesor: & \\
    \cline{2-2}
  \end{tabular}

  \newpage

  \section*{Notes}
  \setlength\parindent{0pt}

  \subsection*{Notes from meeting}

  Sugerencias del profe:
  \begin{enumerate}
    \item 
  \end{enumerate}

  Metas de la semana:
  \begin{enumerate}
    \item Empezar a escribir introducción (ver plantilla) (ver checklist en directorio)
    \item Protocolar perfil de duración de partes (feature extractor, model, etc...) (
      Analizar cambio
    )
  \end{enumerate}

\begin{lstlisting}
  Notas Reunion:
  - Como se hace deteccion?
  - Empezar a escribir introduccion (ver plantilla) (ver checklist en directorio)
  - Protocolar perfil de duracion de partes (feature extractor, model, etc...)
    (Analizar cambio)
  - como se hace deteccion actual?
  - introduccion hablar de baseline
    - modelo de disptec
  en la intro no se mete tecnico, solo dice que existe
  intro diagrama bloques overview
  sistema:
  "aqui esta sistema completo, aqui se ubica sistema mio"
  justificar duracion baseline
  REALMENTE TOMAR EN CUENTA OBSERVACIONES
  capitulo 3
  como se genera set de datos
  12 - abril entrenamiento terminado
  PROCURE TENERLO ANTES
  OpenMP ---

  Wednesday:
  I have fixed the filtering abs diff.
  I have advanced work in border stats calc

  /* FIXME: Refactor question to logging
  * What would total feature file size be?
  * pixels in frame = 1280*720 = 921600
  * blocks in frame = (1280//16)*(720//16) = 3600
  * feature categories per block (borders, diffs, propers)
  * filterings per block = 2
  * borders per filtering = 4
  * rois per border = 4
  * metrics per roi = 2
  * border metrics per block = 2*4*4*2 = 64
  * propers are metrics for block = 2
  * diffs are borders minus propers = 64
  * total 64 + 2 + 64 = 130
  * 4 bytes per feature (floats)
  * bytes per block = 130 * 4 = 520
  * bytes per frame = 3600 * 520 = 1872000 = 1.872 MB
  * frames per snippet 200
  * bytes per snippet = 1872000*200 = 374400000 = 374.4 MB
  * snippets per video 19
  * bytes per video = 374400000*19 = 7113600000 = 7.11 GB
  * videos = 1
  * total dataset feature file size = 7.11 GB
  * This
  */
  /* Filter test */

  /* Log */
  /* FIXME: refactor log into unit*/
  /* I trust these filtered results
   * I must now calculate features.
   * Rectanges at 1,2,4,8 pixels away from each border
   * 
   */

  Thursday:
  I have completed first border calc

  Friday:
  I am considering to refactor the border calc into macros.
  I think I'll be able to make it more maintainable that way.
  Right now it would be harder to maintain.
  Are macros the right abstraction?
  They are famously able to create code with unexpected results.
  However, If I only use them without altering state, perhaps
  They can be useful.

  /* Are recursive macros useful for writting maintainable and
   * efficient loop unrolling?
   *
   * Unrolling decreases branching, which impairs the instr. pipeline
   * However, unrolling increases the instr load.
   *
   * Parallelization perhaps should be considered as well.
   */

   /* Studying code in: https://stackoverflow.com/questions/12447557/can-we-have-recursive-macros
    *
    * Seem like a collection of fun macros.
    * Perhaps they might help improve performance.
    * I should be able to implement it both ways and use profiling to measure
    * Duration.
    */

  /* Studying code in: https://stackoverflow.com/questions/12447557/can-we-have-recursive-macros */
  /* I want to free style the macros */
  // #define CAT(a, ...) PRIMITIVE_CAT(a, __VA_ARGS__)
  // #define PRIMITIVE_CAT(a, ...) a ## __VA_ARGS__
  
  // #define CHECK_N(x, n, ...) n
  // #define CHECK(...) CHECK_N(__VA_ARGS__, 0,) /* One or two args may be used, it seems to me that this would always return the second arg*/

  // #define NOT(x) CHECK(PRIMITIVE_CAT(NOT_, x))
  // #define NOT_0 ~, 1,

  // #define COMPL(b) PRIMITIVE_CAT(COMPL_, b)
  // #define COMPL_1 0
  // #define COMPL_0 1

  // #define BOOL(x) COMPL(NOT(x))

  // #define IIF(c) PRIMITIVE_CAT(IFF_, c)
  // #define IIF_1(c)(t, ...) t
  // #define IIF_0(c)(t, ...) __VA_ARGS__

  // #define IF(c) IIF(BOOL(C))

  // #define EAT(...)
  // #define EXPAND(...) __VA_ARGS__
  // #define WHEN(c) IF(c)(EXPAND, EAT)
  // #define SUM_BLOCK_ROWS_AT_PIC(dst, src, step, id0, ...) \
  //     WHEN(COUNT)


  // WHEN()

  I'm going to try to use other macro tecnique at beginning of forum link.

  // #define EXPAND(...) __VA_ARGS__
  
  // #define EMPTY()
  // #define DEFER(...) __VA_ARGS__ EMPTY()
  
  // #define pr_id() pr
  // #define pr(n) ((n==1)? 1 : DEFER(pr_id)()(n-1))
  


  /* pr(5)
    * expands to ((5==1)? 1 : DEFER(pr_id)()(5-1))
    * ---------------------------------------------
    * would expand to ((5==1)? 1 : pr_id EMPTY()()(5-1))
    * would expand to ((5==1)? 1 : pr_id ()(5-1))
    * would expand to ((5==1)? 1 : pr(5-1))
    * Wouldn't it be marked blue here?
    * 
    * EXPAND(pr(5))
    * expands to pr(5)
    */

    /* I wish to define PIC in this context as Ptr Index Coordinates
    * Pointer Index Coordinates are used to identify row to sum
    */
  /** @brief SUM_BLOCK_ROWS_AT_PIC
    * 
    * 
    */
  Status:
  I tested filtered results against python filtered results.
  I implented macros to aid maintainability while hopefully not causing a performance penalty
  I might want to learn more about recursive macros
  This might allow the creation of a parameter 

  Saturday:
  Status:
  I started to compare python results with c++ results
  Features seemed to give different values.
  I started to print whole sections of memory regions and corresponding numpy buffers,
  These gave different values.
  I searched for implementation errors.

  Sunday:
  /* Thought:
   *
   * I may use my computer use history
   * as a record of line of thoughtn
   * 
   * Example:
   * I just decided to take the laptop upon my lap.
   * This meant that they keyboard layout (KBLO) physically changed
   * This means I changed my software (SW) KBLO
   * Can use of laptop predict keyboard layout?
   * It is one of four factors that predicts keyboard layout
   * 1. Whether I'm using ...
   * Quick interrupting thought:
   *  Spotify should have a record playlist button
   *  I want to try to build one
   *  I want my laptop to run a daemond that allows
   *  customization of every window while keeping the old one
   *  First in simple ways,
   *  I'll see if I can make one that places a button where
   *  it thinks it's most suited to the button's function
   *  Then the button's function will become interfacing
   *  Through the allowed HW permissions
   *  In linux may apps are open source
   *  That is one way to enhance HW permissions
   *  I can think of one hack that could possibly sample
   *  mouse location, use an abstracted (through resizing)
   *  Can I abstract cpp_tools to a gstreamer element?
   *  Could this facilitate conversion of memory blocks to files?
   *  Cheaper Implementation Alternative:
   *  (Gstreamer elements are useful for clients,
   *   which could turn this project into a series of
   *   image processing tools that facilitates the design of
   *   these kinds of experiments,
   *   image an element that recieves 2 streams of buffers,
   *   1. the feature data
   *   2. the label data
   *   3. processes data into trained tree
      Then another element could read trained tree and execute it
      (I already have the element that can execute a trained tree from a file)
      This will be usefull when working whith gstreamer
      How good idea to create threads from Gstreamer element?
      I could at least make sure the executing element works
   */

   Ok, current tasks:
   Modularize code a bunch
   Use template classes?
   I'll try normal class first
   And if I can identify whether the function call is slow
   I'll then try template class
   
   1. Class Design:
   First, I'll, explain the proyect in terms of a block diagram:

   1.1. Final product design
   1.1.1. TFG product
   - Goal: Train Forest
     Mode: Using C++ API
      - Goal: Read lossy frames
        Input  Mode: Video File
        Mode: File read mode:
        - Read using gstreamer API (already implemented)
        - Read using C++ API (not priority)
      - Goal: Create detection mask output
      
        Output Mode: Video File

   - Goal: Feature extraction
     Mode: Original design
    
   Mode :


   -2 
   1.1.2. Disptec product
   -1 Block Image source
    Mode network
    python
   -2
  
  Tuesday:
  - Goal: Modularize into functions
  - Factorized filtering into function
  
  Call stack:
  generate_dataset:
    inputs: video_path, labels_path
    outputs: feature_mem, label_mem
    brief: Iterates over 200 vid frames and for each calls:
      generate_frame_feautures:
        inputs: frame_mem, frame_width, frame_height
        outputs: feature_mem
        brief: Runs filter_frame. Iterates over macroblocks and runs:
          generate_block_features:
            inputs: frame_mem, block_offset, frame_width, frame_height
            outputs: feature_mem
            brief: Calls vertical border func. for u and d
                   Calls horizon. border func. for l and r
                   Calls block func.
                   Calcs diffs
          



          filter_frame:
            inputs: frame_mem, frame_width, frame_height
            outputs: yframe_mem, xframe_mem
        
  Quick profiling:
  
  Python feature generation:
  - Filtering per 200 frames: 4.717424 s
  - Filtering per frame:      0.023587 s =   23.587    ms
  - Features per macroblock:  0.001076 s =    1.076335 ms
  - Features per frame:       3.610027 s = 3610.027    ms
  - Total per frame:          3.633614 s = 3633.614    ms

  C++ feature generation:
  - Filtering per frame:      0.007642 s =    7.642    ms
  - Features per macroblock:  0.000068 s =    0.068    ms 
  - Features per frame:       0.228072 s =  228.072    ms


\end{lstlisting}



  \printbibliography[title={Bibliografía},heading=bibintoc]
\end{document}
