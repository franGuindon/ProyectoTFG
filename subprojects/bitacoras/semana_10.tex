\documentclass[12pt,oneside]{book}
\usepackage[letterpaper, total={19cm, 20cm}]{geometry}

\usepackage{graphicx}
\usepackage{csquotes}
\usepackage[backend=biber,              % Use biber/biblatex
            style=ieee,
            sorting=none,
            citestyle=numeric-comp]{biblatex}
\usepackage{array, multirow}
\usepackage{caption}
\usepackage{subcaption}
\usepackage{amsmath}

\addbibresource{../tesis/literatura.bib}
\usepackage{listings}
\usepackage{xcolor}

\definecolor{codegreen}{rgb}{0,0.6,0}
\definecolor{codegray}{rgb}{0.5,0.5,0.5}
\definecolor{codepurple}{rgb}{0.58,0,0.82}
\definecolor{backcolour}{rgb}{0.95,0.95,0.92}

\lstdefinestyle{mystyle}{
    backgroundcolor=\color{backcolour},   
    commentstyle=\color{codegreen},
    keywordstyle=\color{magenta},
    numberstyle=\tiny\color{codegray},
    stringstyle=\color{codepurple},
    basicstyle=\ttfamily\footnotesize,
    breakatwhitespace=false,         
    breaklines=true,                 
    captionpos=b,                    
    keepspaces=true,                 
    numbers=left,                    
    numbersep=5pt,                  
    showspaces=false,                
    showstringspaces=false,
    showtabs=false,                  
    tabsize=2,
    mathescape=true
}

\lstset{style=mystyle}

% TODO: Add documentation and change name of doc to the effect of .bash_aliases

\usepackage{enumitem} % nested enumerations


\begin{document}
 \graphicspath{{./}{../tesis/fig/}}
  Tecnológico de Costa Rica
  \par\vspace{1mm}
  Escuela de Ingeniería Electrónica
  \par\vspace{1mm}
  Programa de Licenciatura en Ingeniería Electrónica
  \par\vspace{10mm}
  Trabajo Final de Graduación
  \par\vspace{1mm}
  Francis Guindon Badilla
  \par\vspace{1mm}
  2018259419
  \par\vspace{10mm}
  \large\textbf{Bitácora - Semana 10}
  \par\vspace{10mm}
  \small

  \begin{table} [!h]
    \centering
    \small
    \begin{tabular}{p{1.5 cm} p{2.1 cm} p{5 cm} p{8 cm}}
      \hline
      Fecha & Duración & Actividad & Descripción \\
      \hline
      26/4/23 & 0.5 h & Reunión de avance 10 & Metas son: Probar serie de experimentos para verificar datos \\
      26/4/23 & 2 h & Arreglando herramienta de entreno & Buscando error en cómo se entregan datos a ranger \\
      30/4/23 & 6 h & & Se encontró y corrigió en error (se entregaban datos en row major y ranger los recibe en column major) \\
      1/5/23 & 5 h &  Entrenando bosque & Se entrenó denuevo y se obtuvo buenas estadísticas. Se entrenó en Amdahl \\
      2/4/23 & 5.5 h & Tesis & Se avanzó con la escritura de la introducción \\
      3/4/23 & 2 h & Prueba de bosque & Se avanzó con capturar segundo video de prueba \\
      \hline
      \textbf{Total} & 21 h \\
      \hline
    \end{tabular}
  \end{table}
  
  \vfill

  \begin{tabular}{p{3 cm} p{10 cm}}
    Firma profesor: & \\
    \cline{2-2}
  \end{tabular}

  \newpage

  \section*{Notes}
  \setlength\parindent{0pt}

\subsection*{Wednesday:}
\begin{lstlisting}
Meeting:
- Document algorithm (try bingchat to generate text from outline)
- Experiments to try:
- Unit test for dataset
- Perhaps try dummy dataset with black macroblocks and white background
- Perhaps try black macroblocks in image
- Try 9 raw macroblock features
- Add date to result files (at beginning of filename)
\end{lstlisting}

\subsubsection{How does Ranger train?}
To create and train a Ranger forest, there are tree steps:
\begin{enumerate}
  \item Forest construction
  \item Initialization method
  \item Run method
\end{enumerate}

The constructor takes no arguments. During construction, internal members are set to default values.

The initialization method is more complex. It takes many arguments. However it will be addressed in so far as it helps explain the run method.

The run method takes a verbose boolean and a compute oob error boolean. It is responsible for training or predicting, depending on one of the intialization arguments.

\subsubsection{The run method}

\begin{enumerate}
  \item First, Ranger checks the \lstinline|prediction_mode| member.
  \item If Ranger is predicting, it runs the \lstinline|predict()| method.
  \item If Ranger is not predicting, it runs the \lstinline|grow()| method.
  \item Ranger might also run the \lstinline|computePredictionError()| and the \lstinline|computePermutationImportance()| method.
\end{enumerate}

\subsubsection{The grow method}

\begin{enumerate}
  \item First, Ranger runs the \lstinline|equalSplit()| to setup the \lstinline|thread_ranges| variable.
  \item It then calls the \lstinline|growInternal()| method. The internal methods are implemented by the specific Forest subclass being used (\lstinline|ForestClassification| in this case).
  \item Ranger then runs a loop over the number of trees in the forest.
  \item In the loop, it runs some logic to set up the random seed for the tree.
  \item It then runs some logic to set up the \lstinline|tree_split_select_weights| variable.
  \item It then runs some logic to set up the \lstinline|tree_manual_inbag| variable.
  \item It then runs the tree's \lstinline|init| method, which takes several arguments.
  \item After the loop, the \lstinline|variable_importance| vector is resized to \lstinline|num_independent_variables|
  \item Then, \lstinline|num_thread| threads are allocated.
  \item Then, the \lstinline|variable_importance_threads| is created;
  \item Then, Ranger loops over \lstinline|num_threads|.
  \item In each iteration it runs the \lstinline|growTreesInThread| method with the thread id (iteration counter) and the corresponding \lstinline|variable_importance_threads| vector as arguments.
  \item Then, Ranger loops over the threads to join them.
  \item Then, Ranger sums the variable importance values per thread.
  \item Then, Ranger divides each variable importance value by the number of trees.
\end{enumerate}

\subsubsection{The growInternal method}

\begin{enumerate}
  \item The growInternal method constructs each tree.
  \item Each tree takes several arguments \lstinline|class_values, response_classIDs, sampleIDs_per_class, class_weights|
\end{enumerate}

\subsubsection{The growTreesInThread method}

\begin{enumerate}
  \item The tree \lstinline|grow| method is called for every tree assigned to the given thread.
  \item The tree \lstinline|grow| method takes the variable importance as argument.
\end{enumerate}

\subsubsection{The tree grow method}

\begin{enumerate}
  \item \lstinline|allocateMethod| is called.
  \item Then, Ranger runs the bootstrapping logic.
  \item Then, Ranger runs the node splitting logic.
\end{enumerate}

\subsubsection{The node splitting logic}

\begin{enumerate}
  \item A \lstinline|num_open_nodes| variable keeps track of the number of non terminal nodes.
  \item \lstinline|num_open_nodes| is initialized to 1 (corresponding to the root node).
  \item The algorithm stops when \lstinline|num_open_nodes > 0|. In other words, the algorithm stops when all nodes are either inner nodes or terminal nodes.
  \item Each iteration runs runs the \lstinline|splitNode()| method.
  \item This method takes an iteration counter as argument, corresponding to the node id.
  \item This method returns a boolean indicating if the node is terminal.
  \item If the node is terminal, the \lstinline|num_open_nodes| variable is decremented.
  \item If the node is not terminal, the \lstinline|num_open_nodes| variable is incremented. Additionally, if the node id is reaches the leftmost node, the depth counter is increased and the leftmost node id is set to the next layer.
\end{enumerate}

\subsubsection{The splitNode method}

\begin{enumerate}
  \item First, Ranger runs the \lstinline|createPossibleSplitVarSubset()| method.
  \item This method takes a vector of indices as argument, \lstinline|possible_split_var_subset|.
  \item Then, Ranger calls the \lstinline|splitNodeInternal()| method.
  \item This method takes the node id and the possible values dataset as arguments.
  \item This method return a boolean indicating if the node is terminal, which is then returned by the \lstinline|splitNode| method.
  \item Then, the \lstinline|split_varID| and \lstinline|split_value| for the node are defined. This is interesting because perhaps the split values are not being chosen randomly.
  \item 
\end{enumerate}

% \subsubsection{The createPossibleSplitVarSubset method}

% \begin{enumerate}
%   \item 
% \end{enumerate}

\subsection*{Sunday}

\subsubsection{Goals}
\begin{enumerate}
  \item Refactor gstreamer videofilesrc and videofilesink
  \item Generate black and white dataset
  \item Train with black and white dataset
\end{enumerate}

\subsubsection{Results}
\begin{enumerate}
  \item Results from training with black and white dataset
\begin{lstlisting}
python3 ProyectoTFG/tools/python_tools/parse_ranger_confusion_matrix.py models/loss0.1 vid0 black white test ntree32 maxdepth7 splitruleEXTRATREES randomsplit50 balanced randomized.confusion 
Precision: 0.50
Recall:    0.73
\end{lstlisting}
\end{enumerate}

\subsection*{Monday}

\subsubsection{Goals}
\begin{enumerate}
  \item Debug trainer
\end{enumerate}

\section{Debugging trainer}

\subsection*{Trainer script structure:}
\begin{enumerate}
  \item Argument parser
  \item Forest construction
  \item Forest parameters
  \item Raw dataset loading
  \begin{enumerate}
    \item Dataset loading timer start
    \item Dataset memory size calculations
    \item Dataset memory allocation
    \item Dataset features loading %(quite simple)
          and dataset labels loading %(quite complex, needs improvement)    
    \begin{enumerate}
      \item Loop over argument pairs
      \item Loop calculates argument memory offsets
      \item Loop calculates feature memory offsets
      \item Loop calculates label memory offsets
      \item Loop loads raw feature file and raw label file
      \item Loop processes label file into label file in feature order
    \end{enumerate}
    \item Test for labels loading offset mismatch
    \item Print Dataset Loading Duration
  \end{enumerate}
  \item Generating randomized balanced dataset
  \begin{enumerate}
    \item Data balancing timer start
    \item Balancing dataset
    \begin{enumerate}
      \item Dataset balance measure
      \item Randomizing class datasets separately
      \item Balanced dataset calculation  
    \end{enumerate}
    \item Randomizing balanced dataset
    \begin{enumerate}
      \item Randomized balanced dataset memory size calculation
      \item Randomized balanced dataset memory allocation
      \item Initializing ids in ascending order
      \item Shuffling ids
      \item Loop processes balanced dataset into randomized balanced dataset
      \begin{enumerate}
        \item Iterate i over balanced dataset size
        \item Randomized id calculation
        \item Feature array at randomized balanced dataset memory calculation
        \item Feature array at balanced dataset memory calculation
        \item Copy feature from balanced dataset memory to randomized balanced dataset memory
        \item Copy label from balanced dataset memory to randomized balanced dataset memory
      \end{enumerate}
    \end{enumerate}
    \item Print dataset balancing duration
  \end{enumerate}  
  \item Initializing forest with dataset and arguments
  \begin{enumerate}
    \item Forest init timer start
    \item Forest init call
    \item Print Forest init duration
  \end{enumerate}
  \item Growing forest
  \begin{enumerate}
    \item Forest grow timer start
    \item Forest grow call
    \item Print Forest grow duration
  \end{enumerate}
  \item Printing results
\end{enumerate}

\subsection*{Sunday}
Testing debugged method in amdahl
\begin{enumerate}
  \item 
\end{enumerate}

\subsection*{Monday}
\begin{enumerate}
  \item Testing Amdahl with complete dataset and recent bugfixes in features
  \item Changing mtry to 132. This should mean the Forest now requires every node to test every feature for the best split.
  \item Confusion matrix:
  \begin{lstlisting}
    cat 
    models/05_01_2023_07\:47\:07_
    loss0.1_vid0_debugged_
    ntree32_maxdepth10_splitruleEXTRATREES_randomsplit50_mtry_132
    _balanced0.5pos_randomized.confusion 

    Overall OOB prediction error (Fraction missclassified): 0.000742338

    Class specific prediction errors:
                    255     0
    predicted 255     3717304 5512  
    predicted 0     7     3711800
  \end{lstlisting}
  \item Precision and Recall:
  \begin{lstlisting}
    Precision: 1.00
    Recall:    1.00
  \end{lstlisting}
  \item Feature use:
  \begin{lstlisting}
Total tree:
Feature id: 120, count: 1
Feature id: 43, count: 1
Feature id: 8, count: 1
Feature id: 77, count: 1
Feature id: 51, count: 1
Feature id: 89, count: 1
Feature id: 30, count: 3
Feature id: 15, count: 1
Feature id: 92, count: 1
Feature id: 94, count: 1
Feature id: 101, count: 1
Feature id: 73, count: 1
Feature id: 62, count: 1
Feature id: 64, count: 2
Feature id: 74, count: 1
Feature id: 13, count: 1
Feature id: 12, count: 1
Feature id: 116, count: 2
Feature id: 115, count: 3
Feature id: 91, count: 4
Feature id: 84, count: 3
Feature id: 109, count: 1
Feature id: 79, count: 10
Feature id: 123, count: 3
Feature id: 37, count: 1
Feature id: 65, count: 1
Feature id: 67, count: 2
Feature id: 78, count: 2
Feature id: 122, count: 1
Feature id: 4, count: 3
Feature id: 63, count: 18
Feature id: 102, count: 3
Feature id: 52, count: 5
Feature id: 111, count: 13
Feature id: 22, count: 8
Feature id: 34, count: 10
Feature id: 103, count: 21
Feature id: 44, count: 1
Feature id: 87, count: 23
Feature id: 28, count: 4
Feature id: 99, count: 3
Feature id: 40, count: 2
Feature id: 70, count: 7
Feature id: 11, count: 2
Feature id: 117, count: 3
Feature id: 59, count: 5
Feature id: 118, count: 4
Feature id: 97, count: 1
Feature id: 38, count: 5
Feature id: 39, count: 19
Feature id: 98, count: 1
Feature id: 55, count: 22
Feature id: 85, count: 7
Feature id: 7, count: 17
Feature id: 125, count: 2
Feature id: 75, count: 6
Feature id: 46, count: 2
Feature id: 71, count: 28
Feature id: 129, count: 1
Feature id: 2, count: 27
Feature id: 130, count: 1
Feature id: 3, count: 1
Feature id: 119, count: 12
Feature id: 60, count: 1
Feature id: 21, count: 8
Feature id: 0, count: 9
Feature id: 127, count: 2
Feature id: 68, count: 1
Feature id: 6, count: 7
Feature id: 35, count: 29
Feature id: 83, count: 11
Feature id: 53, count: 4
Feature id: 5, count: 8
Feature id: 45, count: 2
Feature id: 33, count: 6
Feature id: 76, count: 2
Feature id: 69, count: 6
Feature id: 23, count: 13
Feature id: 57, count: 1
Feature id: 95, count: 6
Feature id: 36, count: 7
Feature id: 107, count: 4
Feature id: 86, count: 8
Feature id: 47, count: 2
Feature id: 20, count: 4
Feature id: 29, count: 1
Feature id: 113, count: 1
Feature id: 54, count: 3
Feature id: 31, count: 4
Feature id: 14, count: 2
  \end{lstlisting}
\end{enumerate}

\subsection*{Tuesday/Wednesday}

Goals:
\begin{enumerate}
  \item Finish Intro.
  \item Train with another video
  \item Test model0 with another video
  \item Test model* with other videos
  \item Train model with 2 videos
  \item Test that on other videos
\end{enumerate}

Summary:
\begin{enumerate}
  \item Advancing with thesis
  \item Updating git a bit
  \item Generating webcam captures
  \begin{lstlisting}
$\mbox{\textdollar}$ GST_DEBUG=WARNING gst-launch-1.0 -v v4l2src ! xvimagesink
    Setting pipeline to PAUSED ...
    Pipeline is live and does not need PREROLL ...
    0:00:00.098898291 50982 0x55c5f3cdd180 WARN                 basesrc gstbasesrc.c:3072:gst_base_src_loop:<v4l2src0> error: Internal data stream error.
    0:00:00.098955929 50982 0x55c5f3cdd180 
    WARN                 basesrc gstbasesrc.c:3072:gst_base_src_loop:<v4l2src0> error: streaming stopped, reason not-negotiated (-4)
    Setting pipeline to PLAYING ...
    0:00:00.099187668 50982 0x55c5f3cdc460 WARN                 v4l2src gstv4l2src.c:695:gst_v4l2src_query:<v4l2src0> Can't give latency since framerate isn't fixated !
    ERROR: from element /GstPipeline:pipeline0/GstV4l2Src:v4l2src0: Internal data stream error.
    Additional debug info:
    gstbasesrc.c(3072): gst_base_src_loop (): /GstPipeline:pipeline0/GstV4l2Src:v4l2src0:
    streaming stopped, reason not-negotiated (-4)
    Execution ended after 0:00:00.000113851
    Setting pipeline to NULL ...
    Freeing pipeline ...
  \end{lstlisting}
  \item Hipothesis: This error is caused because v4l2src and xvimagesink don't have compatible formats
  \item A quick inspect shows that v4l2src is compatible with xvimagesink:
  \begin{lstlisting} % $\mbox{\textdollar}$

    $\mbox{\textdollar}$ gst-inspect-1.0 v4l2src
      video/x-raw
        format: { (string)RGB16, (string)BGR, (string)RGB, (string)GRAY8, (string)
          GRAY16_LE, (string)GRAY16_BE, (string)YVU9, (string)YV12, (string)YUY2,
          (string)YVYU, (string)UYVY, (string)Y42B, (string)Y41B, (string)YUV9,
          (string)NV12_64Z32, (string)NV24, (string)NV61, (string)NV16, (string)
          NV21, (string)NV12, (string)I420, (string)BGRA, (string)BGRx, (string )
          ARGB, (string)xRGB, (string)BGR15, (string)RGB15 }
    $\mbox{\textdollar}$ gst-inspect-1.0 xvimagesink  
      video/x-raw
        framerate: [ 0/1, 2147483647/1 ]
        width: [ 1, 2147483647 ]
        height: [ 1, 2147483647 ]
      \end{lstlisting}
  \item Found useful pipeline:
  \begin{lstlisting} % $\mbox{\textdollar}$

    $\mbox{\textdollar}$ GST_DEBUG=WARNING gst-launch-1.0 -ve v4l2src name=cam_src num-buffers=200 ! videoconvert ! tee name=t ! queue ! videoscale ! video/x-raw,format=RGB ! queue max-size-buffers=30 ! videoconvert ! ximagesink sync=false name=img_origin t. ! queue max-size-buffers=30 ! 'video/x-raw,framerate=10/1,format=RGB,height=720,width=1280' ! videoconvert ! 'video/x-raw,framerate=10/1,format=I420,height=720,width=1280' ! multifilesink location= /media/francis/Seagate\ Basic/temp200frames/%d.i420
  \end{lstlisting}
  \item Taking advantage of my new 2 TB external drive, I saved 200 raw i420 frames from the webcam
  \item 
\end{enumerate}

% \begin{matrix}
%   1 & 1 & 1 \\
% \end{matrix} $&=$ \begin{matrix}
%   1 & 1 & 1 \\
% \end{matrix}

  \printbibliography[title={Bibliografía},heading=bibintoc]
\end{document}
