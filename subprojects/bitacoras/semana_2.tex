\documentclass[12pt,oneside]{book}
\usepackage[letterpaper, total={19cm, 20cm}]{geometry}

\usepackage{graphicx}
\usepackage{csquotes}
\usepackage[backend=biber,              % Use biber/biblatex
            style=ieee,
            sorting=none,
            citestyle=numeric-comp]{biblatex}
\usepackage{array, multirow}
\usepackage{caption}
\usepackage{subcaption}

\addbibresource{../tesis/literatura.bib}
\usepackage{listings}
\usepackage{xcolor}

\definecolor{codegreen}{rgb}{0,0.6,0}
\definecolor{codegray}{rgb}{0.5,0.5,0.5}
\definecolor{codepurple}{rgb}{0.58,0,0.82}
\definecolor{backcolour}{rgb}{0.95,0.95,0.92}

\lstdefinestyle{mystyle}{
    backgroundcolor=\color{backcolour},   
    commentstyle=\color{codegreen},
    keywordstyle=\color{magenta},
    numberstyle=\tiny\color{codegray},
    stringstyle=\color{codepurple},
    basicstyle=\ttfamily\footnotesize,
    breakatwhitespace=false,         
    breaklines=true,                 
    captionpos=b,                    
    keepspaces=true,                 
    numbers=left,                    
    numbersep=5pt,                  
    showspaces=false,                
    showstringspaces=false,
    showtabs=false,                  
    tabsize=2,
    mathescape=true
}

\lstset{style=mystyle}

% TODO: Add documentation and change name of doc to the effect of .bash_aliases

\usepackage{enumitem} % nested enumerations



\begin{document}
 \graphicspath{{./}{../tesis/fig/}}
  Tecnológico de Costa Rica
  \par\vspace{1mm}
  Escuela de Ingeniería Electrónica
  \par\vspace{1mm}
  Programa de Licenciatura en Ingeniería Electrónica
  \par\vspace{10mm}
  Trabajo Final de Graduación
  \par\vspace{1mm}
  Francis Guindon Badilla
  \par\vspace{1mm}
  2018259419
  \par\vspace{10mm}
  \large\textbf{Bitácora - Semana 2}
  \par\vspace{10mm}
  \small

  \begin{table} [!h]
    \centering
    \small
    \begin{tabular}{p{1.5 cm} p{2.1 cm} p{5 cm} p{8 cm}}
      \hline
      Fecha & Duración & Actividad & Descripción \\
      \hline
      15/2/23 & 1 h & Reunión de avance 2 & Metas: Completar conjunto de datos, Montar ejemplo de ranger, Avanzar con creación de vectores de características \\
      18/2/23 & 4 h &  Montar ejemplo Ranger & Se clonó y construyó Ranger. Se escribió un script de python para fabricar datasets \\
      & 2 h & & Se entrenó ranger sobre un dataset fabricado. Se tomó métrica de velocidad. Se empezó a replicar en jetson \\
      19/2/23 & 2 h & & Se replicó resultados en jetson (es más lento, pero aún así es rápido). \\
      20/2/23 & 3 h & & Estudiando implementación API de Ranger. Parece que hay que hacerle unos cuantos ajustes al API ya que la clase Forest asume que uno va a pasarle archivos de datos y yo quiero pasarle la memoria directamente, pero no parecen ser ajustes complejos. \\
      21/2/23 & 5 h & Avanzar con creación de vectores de características & Adaptando la herramienta de generación de ground truth de Greivin para generar el conjunto de datos de entrenamiento \\
      22/2/23 & 4 h & & Se escribió un script de python para traducir datos de ground truths a dataset para ranger \\
      \hline
      \textbf{Total} & 21 h \\
      \hline
    \end{tabular}
  \end{table}
  
  \vfill

  \begin{tabular}{p{3 cm} p{10 cm}}
    Firma profesor: & \\
    \cline{2-2}
  \end{tabular}

  \newpage

  \section{Notas}

Primero cloné y construí el repositorio:
\begin{lstlisting}
  git clone https://github.com/imbs-hl/ranger.git
  cd cpp_version
  mkdir build
  cd build
  cmake ..
  make \end{lstlisting}

En el directorio aparece un binario.
Para aprender cómo usarlo:
\begin{lstlisting}
  ./ranger --help \end{lstlisting}
  
La salida:
\begin{lstlisting}
  Usage: 
      ./ranger [options]
  
  Options:
      --help                        Print this help.
      --version                     Print version and citation information.
      --verbose                     Turn on verbose mode.
      --file FILE                   Filename of input data. Only numerical values are supported.
      --treetype TYPE               Set tree type to:
                                    TYPE = 1: Classification.
                                    TYPE = 3: Regression.
                                    TYPE = 5: Survival.
                                    (Default: 1)
      --probability                 Grow a Classification forest with probability estimation for the classes.
                                    Use in combination with --treetype 1.
      --depvarname NAME             Name of dependent variable. For survival trees this is the time variable.
      --statusvarname NAME          Name of status variable, only applicable for survival trees.
                                    Coding is 1 for event and 0 for censored.
      --ntree N                     Set number of trees to N.
                                    (Default: 500)
      --mtry N                      Number of variables to possibly split at in each node.
                                    (Default: sqrt(p) with p = number of independent variables)
      --targetpartitionsize N       Set minimal node size to N.
                                    For Classification and Regression growing is stopped if a node reaches a size smaller than N.
                                    For Survival growing is stopped if one child would reach a size smaller than N.
                                    This means nodes with size smaller N can occur for Classification and Regression.
                                    (Default: 1 for Classification, 5 for Regression, and 3 for Survival)
      --maxdepth N                  Set maximal tree depth to N.
                                    Set to 0 for unlimited depth. A value of 1 corresponds to tree stumps (1 split).
      --catvars V1,V2,..            Comma separated list of names of (unordered) categorical variables. 
                                    Categorical variables must contain only positive integer values.
      --write                       Save forest to file <outprefix>.forest.
      --predict FILE                Load forest from FILE and predict with new data. The new data is expected in the exact same 
                                    shape as the training data. If the outcome of your new dataset is unknown, add a dummy column.
      --predall                     Return a matrix with individual predictions for each tree instead of aggregated 
                                    predictions for all trees (classification and regression only).
      --predictiontype TYPE         Set type of prediction to:
                                    TYPE = 1: Return predicted classes or values.
                                    TYPE = 2: Return terminal node IDs per tree for new observations.
                                    (Default: 1)
      --impmeasure TYPE             Set importance mode to:
                                    TYPE = 0: none.
                                    TYPE = 1: Node impurity: Gini for Classification, variance for Regression, sum of test statistic for Survival.
                                    TYPE = 2: Permutation importance, scaled by standard errors.
                                    TYPE = 3: Permutation importance, no scaling.
                                    TYPE = 5: Corrected node impurity: Bias-corrected version of node impurity importance.
                                    TYPE = 6: Local (casewise) permutation importance.
                                    (Default: 0)
      --noreplace                   Sample without replacement.
      --fraction X                  Fraction of observations to sample. Default is 1 for sampling with replacement 
                                    and 0.632 for sampling without replacement.
      --splitrule RULE              Splitting rule:
                                    RULE = 1: Gini for Classification, variance for Regression, logrank for Survival.
                                    RULE = 2: AUC for Survival, not available for Classification and Regression.
                                    RULE = 3: AUC (ignore ties) for Survival, not available for Classification and Regression.
                                    RULE = 4: MAXSTAT for Survival and Regression, not available for Classification.
                                    RULE = 5: ExtraTrees for all tree types.
                                    RULE = 6: BETA for regression, only for (0,1) bounded outcomes.
                                    RULE = 7: Hellinger for Classification, not available for Regression and Survival.
                                    (Default: 1)
      --randomsplits N              Number of random splits to consider for each splitting variable (ExtraTrees splitrule only).
      --alpha VAL                   Significance threshold to allow splitting (MAXSTAT splitrule only).
      --minprop VAL                 Lower quantile of covariate distribtuion to be considered for splitting (MAXSTAT splitrule only).
      --caseweights FILE            Filename of case weights file.
      --holdout                     Hold-out mode. Hold-out all samples with case weight 0 and use these for variable 
                                    importance and prediction error.
      --splitweights FILE           Filename of split select weights file.
      --alwayssplitvars V1,V2,..    Comma separated list of variable names to be always considered for splitting.
      --regcoef r1,r2,..            Comma separated list of regularization coefficients (one for all variables or one for each variable).
      --usedepth                    Use node depth for regularization.
      --skipoob                     Skip computation of OOB error.
      --nthreads N                  Set number of parallel threads to N.
                                    (Default: Number of CPUs available)
      --seed SEED                   Set random seed to SEED.
                                    (Default: No seed)
      --outprefix PREFIX            Prefix for output files.
      --memmode MODE                Set memory mode to:
                                    MODE = 0: double.
                                    MODE = 1: float.
                                    MODE = 2: char.
                                    (Default: 0)
      --savemem                     Use memory saving (but slower) splitting mode.
  
  See README file for details and examples.\end{lstlisting}

Hice script para generar datos de prueba. Corrí el siguiente comando, según recomendaciones del README. Corrió exageradamente rápido.

\begin{lstlisting}
  francis@francis-laptop:~/Documents/PleaseCleanUp/TEC/2023_S1/Proyecto/repos/ranger/cpp_version/build$ ./ranger --verbose --file test_data.dat --depvarname y --treetype 1 --ntree 1000 --nthreads 12
  Starting Ranger.
  Loading input file: test_data.dat.
  Growing trees ..
  Computing prediction error ..
  
  Tree type:                         Classification
  Dependent variable name:           y
  Number of trees:                   1000
  Sample size:                       400
  Number of independent variables:   2
  Mtry:                              1
  Target node size:                  1
  Variable importance mode:          0
  Memory mode:                       0
  Seed:                              0
  Number of threads:                 12
  
  Overall OOB prediction error:      0.0025
  
  Saved confusion matrix to file ranger_out.confusion.
  Finished Ranger. \end{lstlisting}

Para aproximar la duración, corrí:
\begin{lstlisting}
  francis@francis-laptop:~/Documents/PleaseCleanUp/TEC/2023_S1/Proyecto/repos/ranger/cpp_version/build$ touch `date +%s.%N`.pts; ./ranger --verbose --file test_data.dat --depvarname y --treetype 1 --ntree 1000 --nthreads 12; touch `date +%s.%N`.pts;
  Starting Ranger.
  Loading input file: test_data.dat.
  Growing trees ..
  Computing prediction error ..
  
  Tree type:                         Classification
  Dependent variable name:           y
  Number of trees:                   1000
  Sample size:                       400
  Number of independent variables:   2
  Mtry:                              1
  Target node size:                  1
  Variable importance mode:          0
  Memory mode:                       0
  Seed:                              0
  Number of threads:                 12
  
  Overall OOB prediction error:      0.0025
  
  Saved confusion matrix to file ranger_out.confusion.
  Finished Ranger. \end{lstlisting}

Tengo estos dos archivos:

\begin{lstlisting}
  francis@francis-laptop:~/Documents/PleaseCleanUp/TEC/2023_S1/Proyecto/repos/ranger/cpp_version/build$ ls *.pts
  1676829641.825818338.pts  1676829641.848423306.pts \end{lstlisting}

Con este comando conseguí:
\begin{lstlisting}
  francis@francis-laptop:~/Documents/PleaseCleanUp/TEC/2023_S1/Proyecto/repos/ranger/cpp_version/build$ NEW=`echo 1676829641.848423306.pts | sed 's/\.pts$//'`
  francis@francis-laptop:~/Documents/PleaseCleanUp/TEC/2023_S1/Proyecto/repos/ranger/cpp_version/build$ OLD=`echo 1676829641.825818338.pts | sed 's/\.pts$//'`
  francis@francis-laptop:~/Documents/PleaseCleanUp/TEC/2023_S1/Proyecto/repos/ranger/cpp_version/build$ echo $NEW-$OLD | bc
  .022604968 \end{lstlisting}

Una velocidad menor a 22.6 ms para entrenar y calcular métricas de precisión.

\begin{lstlisting}
  francis@francis-laptop:~/Documents/PleaseCleanUp/TEC/2023_S1/Proyecto/repos/ranger/cpp_version/build$ cat ranger_out.confusion 
  Overall OOB prediction error (Fraction missclassified): 0.0025
  
  Class specific prediction errors:
                  0     1
  predicted 0     222   1     
  predicted 1     0     177 \end{lstlisting}

I'll replicate steps on jetson to see duration:
\begin{lstlisting}
  francis@francis-laptop:~/Documents/PleaseCleanUp/TEC/2023_S1/Proyecto/repos/ranger/cpp_version/build$ ssh jetson-tec 
  Welcome to Ubuntu 18.04.6 LTS (GNU/Linux 4.9.201-tegra aarch64)
  
    * Documentation:  https://help.ubuntu.com
    * Management:     https://landscape.canonical.com
    * Support:        https://ubuntu.com/advantage
  This system has been minimized by removing packages and content that are
  not required on a system that users do not log into.
  
  To restore this content, you can run the 'unminimize' command.
  
  301 updates can be applied immediately.
  277 of these updates are standard security updates.
  To see these additional updates run: apt list --upgradable
  
  Last login: Sun Feb 19 12:35:58 2023 from 172.21.6.206
  disptec@disptec-jetson:~$ \end{lstlisting}

Measured time in Jetson for train command is 117 ms (using 12 threads).

Possible next steps:
\begin{enumerate}
  \item Poke around ranger API and try to write my own example code
    how does tree object expect input and output?
    can I recycle my test code from last time?
    how can I capitalize on the ranger example?

  \item Work on feature selection:
    \begin{enumerate}[label*=\arabic*.]
      \item Clone disptec repo and poke around status
      \item poke gstreamer element in develop
      \item poke inpainter element in develop (lower priority)
      \item poke training tool in one of the dirty branches
      \item poke Greivin's histograms in one of the dirty branches
      \item create my own histograms
    \end{enumerate}

  \item Fun alternative: Fast build feature files with python to make some
    quick training tests. The disadvantage in the long run is that probably these files would become too large. Maybe not though, I think I should try it out.
\end{enumerate}

First, I'll poke around the disptec-2022 repo
\begin{lstlisting}
  git clone git@gitlab.ridgerun.com:ridgerun/tec/disptec-2022.git
  cd disptec-2022
  git checkout develop
  cd artifact_detection
  meson setup build --prefix=/usr/ # failed because meson was not found
  pip3 show meson # meson is not installed 
  pip3 install meson
  meson setup build --prefix=/usr/ # failed because ninja was not found
  sudo apt-get install python3 python3-pip python3-setuptools python3-wheel ninja-build
  meson setup build --prefix=/usr/ # success
  cd build
  ninja # success
  sudo ninja install # success
  GST_DEBUG=WARNING gst-launch-1.0 videotestsrc ! video/x-raw, width=1920, height=1080, format=I420 ! artifactdetector modelpath=$MODELPATH upscale=true ! xvimagesink # Failed, did not find artifactdetector
  ls /usr/lib/aarch64-linux-gnu/gstreamer-1.0/libgstartifactdetect.so # success
  gst-inspect-1.0 /usr/lib/aarch64-linux-gnu/gstreamer-1.0/libgstartifactdetect.so # failed \end{lstlisting}

Here is the failed message:
\begin{lstlisting}
  (gst-inspect-1.0:15870): GStreamer-WARNING **: 04:59:51.781: Failed to load plugin '/usr/lib/aarch64-linux-gnu/gstreamer-1.0/libgstartifactdetect.so': liblibdetector.so: cannot open shared object file: No such file or directory
  Could not load plugin file: Opening module failed: liblibdetector.so: cannot open shared object file: No such file or directory\end{lstlisting}
    
Never heard of liblibdetector.so, not sure why it is being looked for. Perhaps there is a missing 'silent' dependency to the library I wasn't aware of that I should install (and require in the meson build).

Out of convenience:
\begin{lstlisting}
  git clone git@gitlab.ridgerun.com:ridgerun/it/rr-tools.git
  cd rr-tools
  echo "source $PWD/environment" >> ~/.bashrc
  source ~/.bashrc \end{lstlisting}

For now, I'll leave that issue be and ask people at RR later. I'm going to test Greivin's ground truth tool:
\begin{lstlisting}
  cd disptec-2022/tools/ground_truth
  cat README.md
  \end{lstlisting}

Which produces:
\begin{lstlisting}
  ## Tools usage
  The ground truth tool is used to generate datasets with custom videos. The generated dataset includes:
  
  - Folder with video corrupted images
  - Folder with respective mask of the video corrupted images
  
  To use the tool follow the next steps:
  
  Create and setup the tool project.
  ```
  mkdir build && meson build
  ```
  
  Build the executable file to use the tool.
  ```
  ninja -C build
  ```
  
  Setup the directories to store the images.
  
  ```
  mkdir video masks
  ```
  
  Use the tool (the input path should be of a H264 video):
  
  ```
  ./build/getvideo path/to/h264video
  ```
  
  After running the tool, the folders will contain the generated dataset of the given video.  
  \end{lstlisting}

I will explicitly use meson differently, because of personal preference haha:
\begin{lstlisting}
  meson setup build \end{lstlisting}
Compilation will have to wait haha:
\begin{lstlisting}
  francis@francis-laptop:~/Documents/PleaseCleanUp/TEC/2023_S1/Proyecto/$ ssh jetson-tec 
  ssh: connect to host 201.207.53.225 port 2222: No route to host
  kex_exchange_identification: Connection closed by remote host \end{lstlisting}

Back in business:
\begin{lstlisting}
  ninja -C build
  mkdir video masks\end{lstlisting}

I should have some test videos on my machine. I think I'll move them to the project folder:
\begin{lstlisting}
  scp SOS_20220204_122232.MOV VID_20220220_152529.MOV jetson-tec:/home/disptec/proyecto/videos \end{lstlisting}

Time to test:
\begin{lstlisting}
  disptec@disptec-jetson:~/proyecto/disptec-2022/tools/ground_truth (develop=)$ ./build/getvideo ~/proyecto/videos/SOS_20220204_122232.MOV \end{lstlisting}

Which outputs:

\begin{lstlisting}
  (getvideo:18621): GStreamer-CRITICAL **: 06:23:16.897: gst_buffer_unmap: assertion 'GST_IS_BUFFER (buffer)' failed

  (getvideo:18621): GStreamer-CRITICAL **: 06:23:17.152: gst_buffer_unmap: assertion 'GST_IS_BUFFER (buffer)' failed

  (getvideo:18621): GStreamer-CRITICAL **: 06:23:17.248: gst_buffer_unmap: assertion 'GST_IS_BUFFER (buffer)' failed

  (getvideo:18621): GStreamer-CRITICAL **: 06:23:17.266: gst_buffer_unmap: assertion 'GST_IS_BUFFER (buffer)' failed

  (getvideo:18621): GStreamer-CRITICAL **: 06:23:17.374: gst_buffer_unmap: assertion 'GST_IS_BUFFER (buffer)' failed

  (getvideo:18621): GStreamer-CRITICAL **: 06:23:17.687: gst_buffer_unmap: assertion 'GST_IS_BUFFER (buffer)' failed

  (getvideo:18621): GStreamer-CRITICAL **: 06:23:17.728: gst_buffer_unmap: assertion 'GST_IS_BUFFER (buffer)' failed

  (getvideo:18621): GStreamer-CRITICAL **: 06:23:17.772: gst_buffer_unmap: assertion 'GST_IS_BUFFER (buffer)' failed

  (getvideo:18621): GStreamer-CRITICAL **: 06:23:17.876: gst_buffer_unmap: assertion 'GST_IS_BUFFER (buffer)' failed

  (getvideo:18621): GStreamer-CRITICAL **: 06:23:17.987: gst_buffer_unmap: assertion 'GST_IS_BUFFER (buffer)' failed

  (getvideo:18621): GStreamer-CRITICAL **: 06:23:18.041: gst_buffer_unmap: assertion 'GST_IS_BUFFER (buffer)' failed

  (getvideo:18621): GStreamer-CRITICAL **: 06:23:18.105: gst_buffer_unmap: assertion 'GST_IS_BUFFER (buffer)' failed

  (getvideo:18621): GStreamer-CRITICAL **: 06:23:18.286: gst_buffer_unmap: assertion 'GST_IS_BUFFER (buffer)' failed

  (getvideo:18621): GStreamer-CRITICAL **: 06:23:18.394: gst_buffer_unmap: assertion 'GST_IS_BUFFER (buffer)' failed

  (getvideo:18621): GStreamer-CRITICAL **: 06:23:18.487: gst_buffer_unmap: assertion 'GST_IS_BUFFER (buffer)' failed

  (getvideo:18621): GStreamer-CRITICAL **: 06:23:18.567: gst_buffer_unmap: assertion 'GST_IS_BUFFER (buffer)' failed

  (getvideo:18621): GStreamer-CRITICAL **: 06:23:18.599: gst_buffer_unmap: assertion 'GST_IS_BUFFER (buffer)' failed

  (getvideo:18621): GStreamer-CRITICAL **: 06:23:18.663: gst_buffer_unmap: assertion 'GST_IS_BUFFER (buffer)' failed

  (getvideo:18621): GStreamer-CRITICAL **: 06:23:18.769: gst_buffer_unmap: assertion 'GST_IS_BUFFER (buffer)' failed

  (getvideo:18621): GStreamer-CRITICAL **: 06:23:18.931: gst_buffer_unmap: assertion 'GST_IS_BUFFER (buffer)' failed

  (getvideo:18621): GStreamer-CRITICAL **: 06:23:19.012: gst_buffer_unmap: assertion 'GST_IS_BUFFER (buffer)' failed

  (getvideo:18621): GStreamer-CRITICAL **: 06:23:19.817: gst_buffer_get_sizes_range: assertion 'GST_IS_BUFFER (buffer)' failed

  (getvideo:18621): GStreamer-CRITICAL **: 06:23:19.817: gst_buffer_unmap: assertion 'GST_IS_BUFFER (buffer)' failed

  (getvideo:18621): GStreamer-CRITICAL **: 06:23:20.075: gst_buffer_unmap: assertion 'GST_IS_BUFFER (buffer)' failed

  (getvideo:18621): GStreamer-CRITICAL **: 06:23:20.086: gst_buffer_unmap: assertion 'GST_IS_BUFFER (buffer)' failed

  (getvideo:18621): GStreamer-CRITICAL **: 06:23:20.117: gst_buffer_unmap: assertion 'GST_IS_BUFFER (buffer)' failed

  (getvideo:18621): GStreamer-CRITICAL **: 06:23:20.340: gst_buffer_unmap: assertion 'GST_IS_BUFFER (buffer)' failed

  (getvideo:18621): GStreamer-CRITICAL **: 06:23:20.373: gst_buffer_unmap: assertion 'GST_IS_BUFFER (buffer)' failed \end{lstlisting}

\begin{lstlisting}
  disptec@disptec-jetson:~/proyecto/disptec-2022/tools/ground_truth (develop=)$ ls video/
  000.png    00107.png  00114.png  00121.png  00129.png  00136.png  00143.png  0015.png  0022.png  002.png   0037.png  0044.png  0051.png  0059.png  0066.png  0073.png  0080.png  0088.png  0095.png
  00100.png  00108.png  00115.png  00122.png  0012.png   00137.png  00144.png  0016.png  0023.png  0030.png  0038.png  0045.png  0052.png  005.png   0067.png  0074.png  0081.png  0089.png  0096.png
  00101.png  00109.png  00116.png  00123.png  00130.png  00138.png  00145.png  0017.png  0024.png  0031.png  0039.png  0046.png  0053.png  0060.png  0068.png  0075.png  0082.png  008.png   0097.png
  00102.png  0010.png   00117.png  00124.png  00131.png  00139.png  00146.png  0018.png  0025.png  0032.png  003.png   0047.png  0054.png  0061.png  0069.png  0076.png  0083.png  0090.png  0098.png
  00103.png  00110.png  00118.png  00125.png  00132.png  0013.png   00147.png  0019.png  0026.png  0033.png  0040.png  0048.png  0055.png  0062.png  006.png   0077.png  0084.png  0091.png  0099.png
  00104.png  00111.png  00119.png  00126.png  00133.png  00140.png  00148.png  001.png   0027.png  0034.png  0041.png  0049.png  0056.png  0063.png  0070.png  0078.png  0085.png  0092.png  009.png
  00105.png  00112.png  0011.png   00127.png  00134.png  00141.png  00149.png  0020.png  0028.png  0035.png  0042.png  004.png   0057.png  0064.png  0071.png  0079.png  0086.png  0093.png
  00106.png  00113.png  00120.png  00128.png  00135.png  00142.png  0014.png   0021.png  0029.png  0036.png  0043.png  0050.png  0058.png  0065.png  0072.png  007.png   0087.png  0094.png
  disptec@disptec-jetson:~/proyecto/disptec-2022/tools/ground_truth (develop=)$ ls masks/
  000.png    00107.png  00114.png  00121.png  00129.png  00136.png  00143.png  0015.png  0022.png  002.png   0037.png  0044.png  0051.png  0059.png  0066.png  0073.png  0080.png  0088.png  0095.png
  00100.png  00108.png  00115.png  00122.png  0012.png   00137.png  00144.png  0016.png  0023.png  0030.png  0038.png  0045.png  0052.png  005.png   0067.png  0074.png  0081.png  0089.png  0096.png
  00101.png  00109.png  00116.png  00123.png  00130.png  00138.png  00145.png  0017.png  0024.png  0031.png  0039.png  0046.png  0053.png  0060.png  0068.png  0075.png  0082.png  008.png   0097.png
  00102.png  0010.png   00117.png  00124.png  00131.png  00139.png  00146.png  0018.png  0025.png  0032.png  003.png   0047.png  0054.png  0061.png  0069.png  0076.png  0083.png  0090.png  0098.png
  00103.png  00110.png  00118.png  00125.png  00132.png  0013.png   00147.png  0019.png  0026.png  0033.png  0040.png  0048.png  0055.png  0062.png  006.png   0077.png  0084.png  0091.png  0099.png
  00104.png  00111.png  00119.png  00126.png  00133.png  00140.png  00148.png  001.png   0027.png  0034.png  0041.png  0049.png  0056.png  0063.png  0070.png  0078.png  0085.png  0092.png  009.png
  00105.png  00112.png  0011.png   00127.png  00134.png  00141.png  00149.png  0020.png  0028.png  0035.png  0042.png  004.png   0057.png  0064.png  0071.png  0079.png  0086.png  0093.png
  00106.png  00113.png  00120.png  00128.png  00135.png  00142.png  0014.png   0021.png  0029.png  0036.png  0043.png  0050.png  0058.png  0065.png  0072.png  007.png   0087.png  0094.png\end{lstlisting}

This works. It produces images at a $432\times240$ resolution. I did notice something interesting about these images though:

  \begin{figure} [!h]
    \centering
    
    \begin{subfigure}[t]{0.49\textwidth}
      \centering
      \includegraphics[width=\textwidth]{0020_video}
      \caption{Lossy video frame 20}
      \label{fig:1.1.a}
    \end{subfigure}
    \hfill
    \begin{subfigure}[t]{0.49\textwidth}
      \centering
      \includegraphics[width=\textwidth]{0020_mask}
      \caption{Ground truth mask for frame 20}
      \label{fig:1.1.b}
    \end{subfigure}
    
    \caption{Comparison of corresponding lossy frame and ground truth mask generated with Greivin's ground truth tool}
    \label{fig:1.1}
  
  \end{figure}

White means a detected artefact, Black means a non corrupted macroblock.
However, these images don't seem to correspond very well. Also, it seems to me that the artefacts were generated in a higher resolution and then the image was downscaled, and then the masks were generated.

For these reasons, I don't believe I may use the tool as is.
However, I think that with some adjustments, I might be able to use it.

First, I'll make my own `develop' branch to keep track of completed features:
\begin{lstlisting}
  git checkout -b dev_francis_tfg\end{lstlisting}

The first feature I want to complete is to duplicate the tool as is with a new name.
\begin{lstlisting}
  git checkout -b feature/duplicate_ground_truth_tool\end{lstlisting}

I'll name the new tool `detector\_dataset\_generator'. I copied the files. Now I am going to test the tool again to make sure it still works. All seems fine. I will now merge this branch into my `develop'. As a next step, I'll remove the downsampling.

This one proved to be more complicated than I originally imagined. By removing downsampling, I'm getting strange segmentation faults. I'll continue working on removing downsampling later. I have managed to generate some lossy frames and some masks. I will work on a script to translate these into a data file that I might train Ranger on.

  \printbibliography[title={Bibliografía},heading=bibintoc]
\end{document}
