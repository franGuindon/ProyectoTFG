\documentclass[12pt,oneside]{book}
\usepackage[letterpaper, total={19cm, 20cm}]{geometry}

\usepackage{graphicx}
\usepackage{csquotes}
\usepackage[backend=biber,              % Use biber/biblatex
            style=ieee,
            sorting=none,
            citestyle=numeric-comp]{biblatex}
\usepackage{array, multirow}
\usepackage{caption}
\usepackage{subcaption}

\addbibresource{../tesis/literatura.bib}
\usepackage{listings}
\usepackage{xcolor}

\definecolor{codegreen}{rgb}{0,0.6,0}
\definecolor{codegray}{rgb}{0.5,0.5,0.5}
\definecolor{codepurple}{rgb}{0.58,0,0.82}
\definecolor{backcolour}{rgb}{0.95,0.95,0.92}

\lstdefinestyle{mystyle}{
    backgroundcolor=\color{backcolour},   
    commentstyle=\color{codegreen},
    keywordstyle=\color{magenta},
    numberstyle=\tiny\color{codegray},
    stringstyle=\color{codepurple},
    basicstyle=\ttfamily\footnotesize,
    breakatwhitespace=false,         
    breaklines=true,                 
    captionpos=b,                    
    keepspaces=true,                 
    numbers=left,                    
    numbersep=5pt,                  
    showspaces=false,                
    showstringspaces=false,
    showtabs=false,                  
    tabsize=2,
    mathescape=true
}

\lstset{style=mystyle}

% TODO: Add documentation and change name of doc to the effect of .bash_aliases

\usepackage{enumitem} % nested enumerations



\begin{document}
 \graphicspath{{./}{../tesis/fig/}}
  Tecnológico de Costa Rica
  \par\vspace{1mm}
  Escuela de Ingeniería Electrónica
  \par\vspace{1mm}
  Programa de Licenciatura en Ingeniería Electrónica
  \par\vspace{10mm}
  Trabajo Final de Graduación
  \par\vspace{1mm}
  Francis Guindon Badilla
  \par\vspace{1mm}
  2018259419
  \par\vspace{10mm}
  \large\textbf{Bitácora - Semana 8}
  \par\vspace{10mm}
  \small

  \begin{table} [!h]
    \centering
    \small
    \begin{tabular}{p{1.5 cm} p{2.1 cm} p{5 cm} p{8 cm}}
      \hline
      Fecha & Duración & Actividad & Descripción \\
      \hline
      29/3/23 & 0.5 h & Reunión de avance 8 & Metas son: Continuar con extracción de features \\
      29/3/23 & 3 h & Extracción características & Se implemetó cálculo de SAT \\
      3/3/23 & 3 h & & Se implementó script de python para hacer testing de resultados de cpp \\
      6/3/23 & 3 h & & Se completó el cálculo de features para un solo bloque \\
      7/3/23 & 3 h & & Se terminó de probar y generar features para todos los datos \\
      8/3/23 & 3 h & Entrenamiento & Se creó herramienta para entrenar bosques \\
      9/3/23 & 3 h & & Se hizo una serie de pruebas de entrenamiento para intentar mejorar los resultados \\
      10/3/23 & 1 h & & Se agregó una función para imprimir contenidos de bosques entrenados \\
      11/3/23 & 1.5 h & & Se analizó estructura de bosques entrenados para notar características de profundidad y estructura \\
      \hline
      \textbf{Total} & 21 h \\
      \hline
    \end{tabular}
  \end{table}
  
  \vfill

  \begin{tabular}{p{3 cm} p{10 cm}}
    Firma profesor: & \\
    \cline{2-2}
  \end{tabular}

  \newpage

  \section*{Notes}
  \setlength\parindent{0pt}

\begin{lstlisting}
Wednesday:
  Meeting notes:
  - Integral image - SAT (Summed-Area Table) : Possible optimization for sums and feature extraction

  Goal:
  - implement SAT function
  - implement python tool to help verify c++ results

  Status:
  - Finished implementing SAT function
  - Used maximum compilation optimization
  Current duration values (microseconds):
  Filtering duration: 792
  Vertical SAT and SAT2 duration: 2965
  Horizontal SAT and SAT2 duration: 2817

Monday:
  Tasks:
  - Complete testing of filtering with testing
  - Complete SAT implementation with testing

  Status:
  - I checked filtering both on python and c++
  - I found c++ bug in the way that pointer increments where done in loop
  - I am currently checking SAT calculation by comparing python and c++ results
  - Completed SAT check

Thursday:
  Tasks:
  - Complete feature file generation

  Status:
  - Implemented new get_vertical_border_sums
  - Fixed compilation issues
  - Implemented new get_horizontal_border_sums
  - Tested with python results
  - Fixed bottom and right border problems

Friday:
  Tasks:
  - Complete feature file generation

  Status:
  - Generated feature file for single block
  - Tested single block features with python
  - Generated feature file for single block row
  - Tested singel block row features with python
  - Found differences in block 17
  - Tested sat sum vs np sum in python
  - Python sat sum agrees with cpp sat sum
  - Python sat sum disagrees with sum
  - Why?
  - Perhaps overflow?
  - Using python astype does not seem to make a difference
  - Which is correct?
  - Perhaps I can try both methods with a smaller area and find point at which
    results differ
  - Result: It is an overflow in python:
      Filtered results where being used to calculate stats
      Filtered results had np.uint8 type
      Solution: Read filtered results in np.uint8 type, reinterpret as np.uint32 type
  - Testing complete frame feature set
  - Python and cpp results agree
  - Durations (microseconds):
      Filtering duration: 925
      Vertical SAT and SAT2 duration: 5330
      Horizontal SAT and SAT2 duration: 5637
      Feature STATS extraction duration: 3841
  - Total feature extraction duration per frame:
      Frame feature extraction duration: 19629
  - It seems that feature extraction tends to decrease when done for several frames
    at once. Theory: This could be because optimization decides to reuse previously allocated memory.
    Optimization for gstreamer element (kinda obvious, though): Allocate all data at beginning
  - Completed succesful generation of all feature files

Saturday:
Some tasks I would like to work on:
- Complete feature file testing
- Complete training tool
- Refactor feature generation into class structure
- Refactor gstreamer management for frame loading into class structure
- Complete Chapter 1 of Thesis

  Priority task:
  - Complete training tool

  Status:
  - Yesterday I generated all feature files
  - I have tested single frame features extensivelly by comparing value by value
    with Python results. I am confident in these results since my Python and my
    Cpp tool use different calculation methods and still arrive at identical
    results. I would like to somehow extend my python tool to test complete
    feature files. However, perhaps I should pause this task and focus instead
    in training tool.
  - I want to work on organizing the git structure a bit.
  - First I cloned my repo in the Release directory.
  - I built the project, but I believe many tools are missing.
  - I wish to attempt to recreate complete dataset:
    - Video snippets
    - Simulated packet loss snippets (lossy snips)
    - Difference between lossy snips and original snips
    - Discretized differences by blocks (both as snips and as byte files)
    - Feature files
  - I moved all my commits to a feature branch
  - I reset my master branch to the Initial commit
  - I created a Develop branch and added an empty commit:
    The empty commit was a more or less ok solution to keeping the develop root
    at the initial commit. It is probably not the best way, but it works.
  - I merged my current work into the develop branch
  - I created a feature to add the dataset creation tools
  - I created a release script to simplify testing process
  - I added bash script to build ground truths
  - I added python script to parse README file with info on where to build snips
  - I will pause this feature and start a feature for the training tool
  - I managed to create a simple trainer that can load a single feature file
    as well as the corresponding labels. This corresponds to 1 snippet.
    This is about 1 GB of data.
    Training takes several minutes.
    Results of initial trainings:
    
    bash: tools/cpp_tools/trainer/trainer ../dataset/0/00/features.bytes ../dataset/0/00/block.bytes 
    Loading feature file
    Loading label file
    Building label memory
    670800 670800
    Initializing Rangerx
    Loading from memory: 0x7f51c2cf1010 0x7f51c29b1010
    Running Rangerx
    Growing trees ..
    Growing trees.. Progress: 10%. Estimated remaining time: 15 minutes, 9 seconds.
    Computing prediction error ..
    Saving model to file
    Saved forest to file rangerx.forest.
    Writting output
    
    Tree type:                         Classification
    Dependent variable name:           y
    Number of trees:                   10
    Sample size:                       670800
    Number of independent variables:   132
    Mtry:                              11
    Target node size:                  1
    Variable importance mode:          0
    Memory mode:                       0
    Seed:                              0
    Number of threads:                 12
    
    Overall OOB prediction error:      0.227485
    
    Saved confusion matrix to file rangerx.confusion.
    
    bash: ls
    build.ninja  compile_commands.json  meson-info  meson-logs  meson-private  rangerx.confusion  rangerx.forest  subprojects  tools
    
    bash: cat rangerx.confusion 
    Overall OOB prediction error (Fraction missclassified): 0.227485
    
    Class specific prediction errors:
                    0     255
    predicted 0     44679996115 
    predicted 255     54929 66129 

Sunday:
  - I fixed the confusion matrix print:

  Loading feature file
  Loading label file
  Building label memory
  670800 670800
  Initializing Rangerx
  Loading from memory: 0x7fcb92383010 0x7fcb92043010
  Running Rangerx
  Growing trees ..
  Growing trees.. Progress: 10%. Estimated remaining time: 15 minutes, 0 seconds.
  Computing prediction error ..
  Saving model to file
  Saved forest to file rangerx.forest.
  Writting output
  
  Tree type:                         Classification
  Dependent variable name:           y
  Number of trees:                   10
  Sample size:                       670800
  Number of independent variables:   132
  Mtry:                              11
  Target node size:                  1
  Variable importance mode:          0
  Memory mode:                       0
  Seed:                              0
  Number of threads:                 12
  
  Overall OOB prediction error:      0.229189
  
  Saved confusion matrix to file rangerx.confusion.

confusion matrix:

Overall OOB prediction error (Fraction missclassified): 0.229189

Class specific prediction errors:
                0     255
predicted 0     446283 96816 
predicted 255     55329 65413

  - From these values, I can get some initial values for precision and recall:

  - Sample size: 670800
  - True positives: 65413
  - False positives: 55329
  - Precision: 65413/(65413 + 55329) = 0.54

  - False negatives: 96816
  - Recall: 65413/(65413 + 96816) = 0.40

  - This is training with the first snippet of data and without balancing.

I will try training with all the sets:

Video 0 Snip 1:
Overall OOB prediction error (Fraction missclassified): 0.351771

Class specific prediction errors:
                0     255
predicted 0     289744 126971 
predicted 255     106550 140578 

  - Precision: 140578/(140578 + 106550) = 0.57
  - Recall: 140578/(140578 + 126971) = 0.53

Video 0 Snip 2:
Overall OOB prediction error (Fraction missclassified): 0.28627

Class specific prediction errors:
                0     255
predicted 0     363445 108698 
predicted 255     81375 110447

  - Precision: 0.58
  - Recall: 0.50

- I updated tree maxdepth to 10:
Video 0 Snip 0:
francis@francis-laptop:~/Documents/2023_S1/Proyecto/build (feature/support_complete_data_training=)$ tools/cpp_tools/trainer/trainer ../dataset/0/00/features.bytes ../dataset/0/00/block.bytes 
Loading feature file
Loading label file
Building label memory
Initializing Rangerx
Loading from memory: 0x7fa047a44010 0x7fa047704010
Running Rangerx
Growing trees ..
Growing trees.. Progress: 65%. Estimated remaining time: 22 seconds.
Computing prediction error ..
Saving model to file
Saved forest to file rangerx.forest.
Writting output

Tree type:                         Classification
Dependent variable name:           y
Number of trees:                   20
Sample size:                       670800
Number of independent variables:   132
Mtry:                              11
Target node size:                  1
Variable importance mode:          0
Memory mode:                       0
Seed:                              0
Number of threads:                 12

Overall OOB prediction error:      0.244261

Saved confusion matrix to file rangerx.confusion.
francis@francis-laptop:~/Documents/2023_S1/Proyecto/build (feature/support_complete_data_training=)$ cat rangerx.confusion 
Overall OOB prediction error (Fraction missclassified): 0.244261

Class specific prediction errors:
                0     255
predicted 0     506787 163765 
predicted 255     72    122   
francis@francis-laptop:~/Documents/2023_S1/Proyecto/tools/python_tools (feature/support_complete_data_training=)$ python3 parse_ranger_confusion_matrix.py ../../build/rangerx.confusion 
Precision: 0.63
Recall:    0.00

- Now trying maxdepth 100
Video 0 Snip 0:
francis@francis-laptop:~/Documents/2023_S1/Proyecto/build (feature/support_complete_data_training=)$ tools/cpp_tools/trainer/trainer ../dataset/0/00/features.bytes ../dataset/0/00/block.bytes 
Loading feature file
Loading label file
Building label memory
Initializing Rangerx
Loading from memory: 0x7fe5e71a1010 0x7fe5e6e61010
Running Rangerx
Growing trees ..
Growing trees.. Progress: 5%. Estimated remaining time: 39 minutes, 16 seconds.
Growing trees.. Progress: 65%. Estimated remaining time: 1 minute, 54 seconds.
Computing prediction error ..
Saving model to file
Saved forest to file rangerx.forest.
Writting output

Tree type:                         Classification
Dependent variable name:           y
Number of trees:                   20
Sample size:                       670800
Number of independent variables:   132
Mtry:                              11
Target node size:                  1
Variable importance mode:          0
Memory mode:                       0
Seed:                              0
Number of threads:                 12

Overall OOB prediction error:      0.190811

Saved confusion matrix to file rangerx.confusion.
francis@francis-laptop:~/Documents/2023_S1/Proyecto/build (feature/support_complete_data_training=)$ cat rangerx.confusion 
Overall OOB prediction error (Fraction missclassified): 0.190811

Class specific prediction errors:
                0     255
predicted 0     481307 102439 
predicted 255     25546 61451
Precision: 0.71
Recall:    0.37

Monday:
  Some tasks I would like to work on:
  - Complete feature file testing
  - Complete training tool
  - Refactor feature generation into class structure
  - Refactor gstreamer management for frame loading into class structure
  - Complete Chapter 1 of Thesis

  Priority tasks:
  - Adding a method to view tree structures

Tuesday:
  Status:
  - Follow predict function traceback:

Data:
  Forest.cpp: Tree::child_nodeIds

Traceback:
  Forest.cpp: Forest::predict
    Forest.cpp: Forest::predictTreesInThread
      Tree.cpp: Tree::predict

\end{lstlisting}



  \printbibliography[title={Bibliografía},heading=bibintoc]
\end{document}
