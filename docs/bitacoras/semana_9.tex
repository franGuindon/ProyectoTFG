\documentclass[12pt,oneside]{book}
\usepackage[letterpaper, total={19cm, 20cm}]{geometry}

\usepackage{graphicx}
\usepackage{csquotes}
\usepackage[backend=biber,              % Use biber/biblatex
            style=ieee,
            sorting=none,
            citestyle=numeric-comp]{biblatex}
\usepackage{array, multirow}
\usepackage{caption}
\usepackage{subcaption}

\addbibresource{../tesis/literatura.bib}
\usepackage{listings}
\usepackage{xcolor}

\definecolor{codegreen}{rgb}{0,0.6,0}
\definecolor{codegray}{rgb}{0.5,0.5,0.5}
\definecolor{codepurple}{rgb}{0.58,0,0.82}
\definecolor{backcolour}{rgb}{0.95,0.95,0.92}

\lstdefinestyle{mystyle}{
    backgroundcolor=\color{backcolour},   
    commentstyle=\color{codegreen},
    keywordstyle=\color{magenta},
    numberstyle=\tiny\color{codegray},
    stringstyle=\color{codepurple},
    basicstyle=\ttfamily\footnotesize,
    breakatwhitespace=false,         
    breaklines=true,                 
    captionpos=b,                    
    keepspaces=true,                 
    numbers=left,                    
    numbersep=5pt,                  
    showspaces=false,                
    showstringspaces=false,
    showtabs=false,                  
    tabsize=2,
    mathescape=true
}

\lstset{style=mystyle}

% TODO: Add documentation and change name of doc to the effect of .bash_aliases

\usepackage{enumitem} % nested enumerations



\begin{document}
 \graphicspath{{./}{../tesis/fig/}}
  Tecnológico de Costa Rica
  \par\vspace{1mm}
  Escuela de Ingeniería Electrónica
  \par\vspace{1mm}
  Programa de Licenciatura en Ingeniería Electrónica
  \par\vspace{10mm}
  Trabajo Final de Graduación
  \par\vspace{1mm}
  Francis Guindon Badilla
  \par\vspace{1mm}
  2018259419
  \par\vspace{10mm}
  \large\textbf{Bitácora - Semana 9}
  \par\vspace{10mm}
  \small

  \begin{table} [!h]
    \centering
    \small
    \begin{tabular}{p{1.5 cm} p{2.1 cm} p{5 cm} p{8 cm}}
      \hline
      Fecha & Duración & Actividad & Descripción \\
      \hline
      12/4/23 & 0.5 h & Reunión de avance 8 & Metas son: Balancear datos de entrenamiento, entrenar con datos completos en Amdahl \\
      14/4/23 & 4 h & Balanceo de datos & Agregar lógica de balanceo de datos \\
      15/4/23 & 1 h & & Solucionar una pulga con balanceo de datos \\
      15/4/23 & 5.5 h & Entrenamiento & Recrear el dataset en Amdahl y entrenar con todos los snips por separado \\
      16/4/23 & 2 h & & Actualizar herramienta para entrenar para que pueda manejar todos los datos a la vez \\
      16/4/23 & 2 h & & Entrenando varios experimentos sobre los datos completos \\
      17/4/23 & 2 h & & Actualizando funciones para análisis de bosques \\
      17/4/23 & 2 h & & Entrenando nuevos experimentos para probar balance de árboles \\
      18/4/23 & 2 h & & Analizando cómo Ranger escoge características y umbrales en los nodos \\
      \hline
      \textbf{Total} & 21 h \\
      \hline
    \end{tabular}
  \end{table}
  
  \vfill

  \begin{tabular}{p{3 cm} p{10 cm}}
    Firma profesor: & \\
    \cline{2-2}
  \end{tabular}

  \newpage

  \section*{Notes}
  \setlength\parindent{0pt}

\begin{lstlisting}
Wednesday:

Meeting:
- Look into Opencv SAT: \url{https://docs.opencv.org/3.4/d7/d1b/group__imgproc__misc.html#ga97b87bec26908237e8ba0f6e96d23e28}
- (Low priority) Verify labels and features correspond
- Balance data (High priority)
- Idea: Self implemented SAT is justified (as opposed to using opencv) because we can use loop unrolling with filtering to use cache hits more efficiently
- Test dataset balancing ratios
- Number of experiments per node - Figure out how ranger does it and which parameter to adjust
- What randomizations is Ranger using

Friday:
# Tasks:
- Balancing data

# Status:
- 7:34 PM Start
- Commited git work
- Retrain a forest to review command:
tools/cpp_tools/trainer/trainer ../dataset/0/00/features.bytes ../dataset/0/00/block.bytes
cat rangerx.confusion
python3 ../tools/python_tools/parse_ranger_confusion_matrix.py rangerx.confusion
- Add output argument
- Add count of data balance
- Added balancing logic
- I have bug that fills too much RAM and kills program
- I'll leave that for tomorrow
- 10:50 PM End

Saturday:
# Tasks:
- Fix data balancing

# Status
- 8:00 AM Start
- Fixed data balancing (Wrong Forest->initCpp dataset size argument)
- Created script to train all snippets separately
- Completed separate training
- Working on dataset building scripts
- Managed to build dataset from fresh repo clone

Sunday:
- Managed to refactor tfg code greatly
- Managed to update trainer to be able to use complete dataset
- Managed to train on all data

Monday:
- Analyzing results:
\end{lstlisting}

\begin{lstlisting}
Dataset loading duration: 7345256
Measuring raw dataset balance
Negatives: 9027886, Positives: 3717314, Total: 670800, Balance: 0.71 : 0.29
Randomizing negative and positive ids separately
Adjusting dataset balance to 0.50% pos
Balanced neg cnt: 3717314, pos cnt: 3717314
Dataset balancing duration: 6687628
Initializing Rangerx
Loading from memory: 0x7f2a0f44b010 0x7f2a0d7ee010
Rangerx init duration: 73854243
Running Rangerx
Growing trees ..
Growing trees.. Progress: 5%. Estimated remaining time: 12 hours, 46 minutes, 1 seconds.
Growing trees.. Progress: 55%. Estimated remaining time: 33 minutes, 25 seconds.
Computing prediction error ..
Saving model to file
Saved forest to file models/loss0.1_vid0_ntree20_maxdepth100_balanced.forest.
Rangerx training duration: 2501720274
Writting output

Tree type:                         Classification
Dependent variable name:           y
Number of trees:                   20
Sample size:                       7434628
Number of independent variables:   132
Mtry:                              11
Target node size:                  1
Variable importance mode:          0
Memory mode:                       0
Seed:                              0
Number of threads:                 32

Overall OOB prediction error:      0.500361

Saved confusion matrix to file models/loss0.1_vid0_ntree20_maxdepth100_balanced.confusion.
\end{lstlisting}
Confusion matrix:
\begin{lstlisting}
Overall OOB prediction error (Fraction missclassified): 0.500361

Class specific prediction errors:
                0     255
predicted 0     969245 971928 
predicted 255     2747688 2745004
\end{lstlisting}

Metrics
\begin{lstlisting}
Precision: 0.50
Recall:    0.74
\end{lstlisting}

I noticed the tree is not very balanced.

I retrained using 32 trees, 20 maxdepth:

\begin{lstlisting}
  Dataset loading duration: 7632754
  Measuring raw dataset balance
  Negatives: 9027886, Positives: 3717314, Total: 670800, Balance: 0.71 : 0.29
  Randomizing negative and positive ids separately
  Adjusting dataset balance to 0.50% pos
  Balanced neg cnt: 3717314, pos cnt: 3717314
  Dataset balancing duration: 6831589
  Initializing Rangerx
  Loading from memory: 0x7f066e455010 0x7f066c7f8010
  Rangerx init duration: 74144700
  Running Rangerx
  Growing trees ..
  Growing trees.. Progress: 3%. Estimated remaining time: 4 hours, 21 minutes, 26 seconds.
  Computing prediction error ..
  Saving model to file
  Saved forest to file models/loss0.1_vid0_ntree32_maxdepth20_balanced.forest.
  Rangerx training duration: 551824036
  Writting output
  
  Tree type:                         Classification
  Dependent variable name:           y
  Number of trees:                   32
  Sample size:                       7434628
  Number of independent variables:   132
  Mtry:                              11
  Target node size:                  1
  Variable importance mode:          0
  Memory mode:                       0
  Seed:                              0
  Number of threads:                 32
  
  Overall OOB prediction error:      0.50044  
\end{lstlisting}

The metrics really did not change:
\begin{lstlisting}
Precision: 0.50
Recall:    0.74
\end{lstlisting}

The balanced dataset being handed over to ranger is separated by class.
This might have an effect in the balance of the trained trees.

Tuesday:
\subsubsection{How does Ranger train?}
To create and train a Ranger forest, there are tree steps:
\begin{enumerate}
  \item Forest construction
  \item Initialization method
  \item Run method
\end{enumerate}

The constructor takes no arguments. During construction, internal members are set to default values.

The initialization method is more complex. It takes many arguments. However it will be addressed in so far as it helps explain the run method.

The run method takes a verbose boolean and a compute oob error boolean. It is responsible for training or predicting, depending on one of the intialization arguments.

\subsubsection{The run method}

\begin{enumerate}
  \item First, Ranger checks the \lstinline|prediction_mode| member.
  \item If Ranger is predicting, it runs the \lstinline|predict()| method.
  \item If Ranger is not predicting, it runs the \lstinline|grow()| method.
  \item Ranger might also run the \lstinline|computePredictionError()| and the \lstinline|computePermutationImportance()| method.
\end{enumerate}

\subsubsection{The grow method}

\begin{enumerate}
  \item First, Ranger runs the \lstinline|equalSplit()| to setup the \lstinline|thread_ranges| variable.
  \item It then calls the \lstinline|growInternal()| method. The internal methods are implemented by the specific Forest subclass being used (\lstinline|ForestClassification| in this case).
  \item Ranger then runs a loop over the number of trees in the forest.
  \item In the loop, it runs some logic to set up the random seed for the tree.
  \item It then runs some logic to set up the \lstinline|tree_split_select_weights| variable.
  \item It then runs some logic to set up the \lstinline|tree_manual_inbag| variable.
  \item It then runs the tree's \lstinline|init| method, which takes several arguments.
  \item After the loop, the \lstinline|variable_importance| vector is resized to \lstinline|num_independent_variables|
  \item Then, \lstinline|num_thread| threads are allocated.
  \item Then, the \lstinline|variable_importance_threads| is created;
  \item Then, Ranger loops over \lstinline|num_threads|.
  \item In each iteration it runs the \lstinline|growTreesInThread| method with the thread id (iteration counter) and the corresponding \lstinline|variable_importance_threads| vector as arguments.
  \item Then, Ranger loops over the threads to join them.
  \item Then, Ranger sums the variable importance values per thread.
  \item Then, Ranger divides each variable importance value by the number of trees.
\end{enumerate}

\subsubsection{The growInternal method}

\begin{enumerate}
  \item The growInternal method constructs each tree.
  \item Each tree takes several arguments \lstinline|class_values, response_classIDs, sampleIDs_per_class, class_weights|
\end{enumerate}

\subsubsection{The growTreesInThread method}

\begin{enumerate}
  \item The tree \lstinline|grow| method is called for every tree assigned to the given thread.
  \item The tree \lstinline|grow| method takes the variable importance as argument.
\end{enumerate}

\subsubsection{The tree grow method}

\begin{enumerate}
  \item \lstinline|allocateMethod| is called.
  \item Then, Ranger runs the bootstrapping logic.
  \item Then, Ranger runs the node splitting logic.
\end{enumerate}

\subsubsection{The node splitting logic}

\begin{enumerate}
  \item A \lstinline|num_open_nodes| variable keeps track of the number of non terminal nodes.
  \item \lstinline|num_open_nodes| is initialized to 1 (corresponding to the root node).
  \item The algorithm stops when \lstinline|num_open_nodes > 0|. In other words, the algorithm stops when all nodes are either inner nodes or terminal nodes.
  \item Each iteration runs runs the \lstinline|splitNode()| method.
  \item This method takes an iteration counter as argument, corresponding to the node id.
  \item This method returns a boolean indicating if the node is terminal.
  \item If the node is terminal, the \lstinline|num_open_nodes| variable is decremented.
  \item If the node is not terminal, the \lstinline|num_open_nodes| variable is incremented. Additionally, if the node id is reaches the leftmost node, the depth counter is increased and the leftmost node id is set to the next layer.
\end{enumerate}

\subsubsection{The splitNode method}

\begin{enumerate}
  \item First, Ranger runs the \lstinline|createPossibleSplitVarSubset()| method.
  \item This method takes a vector of indices as argument, \lstinline|possible_split_var_subset|.
  \item Then, Ranger calls the \lstinline|splitNodeInternal()| method.
  \item This method takes the node id and the possible values dataset as arguments.
  \item This method return a boolean indicating if the node is terminal, which is then returned by the \lstinline|splitNode| method.
  \item Then, the \lstinline|split_varID| and \lstinline|split_value| for the node are defined. This is interesting because perhaps the split values are not being chosen randomly.
  \item 
\end{enumerate}

% \subsubsection{The createPossibleSplitVarSubset method}

% \begin{enumerate}
%   \item 
% \end{enumerate}

  \printbibliography[title={Bibliografía},heading=bibintoc]
\end{document}
