\documentclass[12pt,oneside]{book}
\usepackage[letterpaper, total={19cm, 20cm}]{geometry}

\usepackage{graphicx}
\usepackage{csquotes}
\usepackage[backend=biber,              % Use biber/biblatex
            style=ieee,
            sorting=none,
            citestyle=numeric-comp]{biblatex}
\usepackage{array, multirow}
\usepackage{caption}
\usepackage{subcaption}

\addbibresource{../tesis/literatura.bib}
\usepackage{listings}
\usepackage{xcolor}

\definecolor{codegreen}{rgb}{0,0.6,0}
\definecolor{codegray}{rgb}{0.5,0.5,0.5}
\definecolor{codepurple}{rgb}{0.58,0,0.82}
\definecolor{backcolour}{rgb}{0.95,0.95,0.92}

\lstdefinestyle{mystyle}{
    backgroundcolor=\color{backcolour},   
    commentstyle=\color{codegreen},
    keywordstyle=\color{magenta},
    numberstyle=\tiny\color{codegray},
    stringstyle=\color{codepurple},
    basicstyle=\ttfamily\footnotesize,
    breakatwhitespace=false,         
    breaklines=true,                 
    captionpos=b,                    
    keepspaces=true,                 
    numbers=left,                    
    numbersep=5pt,                  
    showspaces=false,                
    showstringspaces=false,
    showtabs=false,                  
    tabsize=2,
    mathescape=true
}

\lstset{style=mystyle}

% TODO: Add documentation and change name of doc to the effect of .bash_aliases

\usepackage{enumitem} % nested enumerations



\begin{document}
 \graphicspath{{./}{../tesis/fig/}}
  Tecnológico de Costa Rica
  \par\vspace{1mm}
  Escuela de Ingeniería Electrónica
  \par\vspace{1mm}
  Programa de Licenciatura en Ingeniería Electrónica
  \par\vspace{10mm}
  Trabajo Final de Graduación
  \par\vspace{1mm}
  Francis Guindon Badilla
  \par\vspace{1mm}
  2018259419
  \par\vspace{10mm}
  \large\textbf{Bitácora - Semana 6}
  \par\vspace{10mm}
  \small

  \begin{table} [!h]
    \centering
    \small
    \begin{tabular}{p{1.5 cm} p{2.1 cm} p{5 cm} p{8 cm}}
      \hline
      Fecha & Duración & Actividad & Descripción \\
      \hline
      16/3/23 & 1 h & Reunión de avance 5 & Metas es: Terminar de montar función de Ranger para estar entrenando \\
      16/3/23 & 1 h & Función Ranger & Se logró desacoplar la creación del objeto datos de la función de Ranger \\
      16/3/23 & 3 h & Presentación anteproyecto & Se alistó una presentación para el anteproyecto \\
      17/3/23 & 1 h & Presentación anteproyecto & Se presentó el anteproyecto ante el tribunal \\
      17/3/23 & 3 h & Función Ranger &  \\
      \hline
      \textbf{Total} & 21 h \\
      \hline
    \end{tabular}
  \end{table}
  
  \vfill

  \begin{tabular}{p{3 cm} p{10 cm}}
    Firma profesor: & \\
    \cline{2-2}
  \end{tabular}

  \newpage

  \section*{Notes}
  \setlength\parindent{0pt}

  \subsection*{Notes from meeting}

  Sugerencias del profe:
  \begin{enumerate}
    \item presentación 10 min 6 folios
    \item contexto disptec - cola, brete de Greivin, problema tiene que quedar claro
    \item objetivos y indicadores y entregable
    \item cronograma
    \item Recordar sobre tesis en ingles
  \end{enumerate}

  Metas de la semana:
  \begin{enumerate}
    \item Agregar y entrenar con nuevos features
    \item Terminar funcion de Ranger
    \item Investigar porcentaje comun de perdida en h264 streaming
  \end{enumerate}

  Observaciones de la reunión:
  \begin{enumerate}
    \item Comparar con resultados disptec ¿Métricas de detección-tiempo?
    \item Justificar 70\% en base a lo que se ha logrado
    \item Justificar 30 fps en base de lo que se ha logrado ... ?
    \item Agregar más sobre generación de conjunto de entrenamiento
    \item ¿Cómo se genera y porqué?
    \item ir escribiendo tesis
    \item tener intro lista
    \item diagrama de bloques
    \item Figuras y tablas son subjetos (voz activa)
    \item formalizmos de tec español, maybe portada, 
  \end{enumerate}

\begin{lstlisting}
  Goal: connect to custom init fnc ... done
  Status:
    The creation of data objects during Forest loader has been decoupled
    This allows the replacement of the data loading function
  8:39 19/3/23
  Goal: Add custom data obj loader (to forest)
  8:48
  Breakfast break
  12:46
  Goal: Add custom data obj loader (to forest) ... done
  13:37
  Goal: Add custom data obj loader (to data)
  16:51
  17:51
  Status:
    I have copied the Data.h file to the extension directory.
    I have added the function.
    However, it has not been recognized.
    I suspect it might be related with having 2 Data.h in the namespace
    I am experimenting with changing the file name and such
    Currently no success
  Goal: Identify error source ... done
  Goal: Identify possible solution ... done
  17:53
  Goal: Implement solution (include ext files in main) ... failed :'c
  Idea: Can a file be excluded from an inc dir? Nope
  Goal: Make a new directory within ranger ... done
  Goal: Fixing duplicated namespace error ... done
  19:03
  
\end{lstlisting}

  \printbibliography[title={Bibliografía},heading=bibintoc]
\end{document}
