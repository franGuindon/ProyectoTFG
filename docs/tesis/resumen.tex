\chapter*{Resumen}
\thispagestyle{empty}

El resumen es la síntesis de lo que aparece en el resto del
documento. Tiene que ser lo suficientemente conciso y claro para que
alguien que lo lea sepa qué esperar del resto del trabajo, y se motive
para leerla completamente.  Usualmente resume lo más relevante de la
introducción y contiene la conclusión más importante del trabajo.

Es usual agregar palabras clave, que son los temas principales
tratados en el documento. El resumen queda fuera de la numeración del
resto de secciones.

Evite utilizar referencias bibliográficas, \tablas, o figuras en el
resumen.

\bigskip

%% Defina las palabras clave con defKeywords en config.tex:
\textbf{Palabras clave:} \thesisKeywords

\clearpage
\chapter*{Abstract}
\thispagestyle{empty}

Current advances in network speed, image processing, and digital technology have rendendered live video tools commonplace in professional, academic, and recreational contexts. Therefore, there has been increased interest in ensuring a high quality of experience (QoE) for these services. These services may suffer from video artifacts caused by packet losses during video transmission. The previous dispTEC2-2022 project aimed to create a Video Restoration System to efficiently restore video with losses on-line, but concluded without a proper Artifact Detection System. This project proposes Rangerx, which improves in 3 ways over the dispTEC2-2022 Random Decision Forest (RDF) artifact detection approach: Rangerx was trained with a larger dataset, with a new optimized feature extraction strategy, and with a new optimized RDF backend.

\bigskip

\textbf{Keywords:} word 1, word 2, 

\cleardoublepage

%%% Local Variables: 
%%% mode: latex
%%% TeX-master: "main"
%%% End: 
