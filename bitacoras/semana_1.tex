\documentclass[12pt,oneside]{book}
\usepackage[letterpaper, total={19cm, 20cm}]{geometry}

\usepackage{graphicx}
\usepackage{csquotes}
\usepackage[backend=biber,              % Use biber/biblatex
            style=ieee,
            sorting=none,
            citestyle=numeric-comp]{biblatex}
\usepackage{array, multirow}

\addbibresource{../tesis/literatura.bib}


\begin{document}
 \graphicspath{{./}{../tesis/fig/}}
  Tecnológico de Costa Rica
  \par\vspace{1mm}
  Escuela de Ingeniería Electrónica
  \par\vspace{1mm}
  Programa de Licenciatura en Ingeniería Electrónica
  \par\vspace{10mm}
  Trabajo Final de Graduación
  \par\vspace{1mm}
  Francis Guindon Badilla
  \par\vspace{1mm}
  2018259419
  \par\vspace{10mm}
  \textbf{Bitácora - Semana 1}
  \par\vspace{10mm}

  \begin{table} [!h]
    \centering
    \begin{tabular}{p{1.5 cm} p{1.7 cm} p{5 cm} p{8 cm}}
      \hline
      Fecha & Duración & Actividad & Descripción \\
      \hline
      8/2/23 & 1 h & Reunión de avance 1 & Definir metas de semana 1 \\
      9/2/23 & 4 h & Reunión con Prof. Miguel Hernandez & Discutir generalidades del TFG \\
      9/2/23 & 3 h & Anteproyecto & Implementar correcciones a proyecto \\
      10/2/23 & 5 h & Investigación Artefactos & Investigar características principales de los artefactos de video \cite{Greengrass2009,Glavota2016} \\
      13/2/23 & 4 h & Investigación en Bosques de Búsqueda Aleatoria & Estudiar matemática y algoritmos involucrados en los RDF. Estudiar marcos de trabajo de RDF. \\
      14/2/23 & 4 h & Definición de características a utilizar & Definir características de entrenamiento a utilizar según las propiedades estadísticas de los artefactos de video. \\
      \hline
      \textbf{Total} & 21 h \\
      \hline
    \end{tabular}
  \end{table}
  
  \vfill

  \begin{tabular}{p{3 cm} p{10 cm}}
    Firma profesor: & \\
    \cline{2-2}
  \end{tabular}

  \printbibliography[title={Bibliografía},heading=bibintoc]
\end{document}
